% \iffalse meta-comment
%
% File: se2packages.dtx Copyright (C) 2022--2024 Stephan Lukasczyk
%
% It may be distributed and/or modified under the conditions of the
% LaTeX Project Public License (LPPL), either version 1.3c of this
% license or (at your option) any later version.  The latest version
% of this license is in the file
%
%   https://www.latex-project.org/lppl.txt
%
% This file is part of the "se2thesis bundle" (The Work in LPPL)
% and all files in that bundle must be distributed together.
%
% The released version of this bundle is available from CTAN.
%
% ----------------------------------------------------------------------
%
% The development version of the bundle can be found at
%
%   https://github.com/se2p/se2thesis
%
% for those people who are interested.
%
% ----------------------------------------------------------------------
%
%<*driver>
\documentclass{l3doc}
% The next line is needed so that \GetFileInfo will be able to pick up
% version data.
\usepackage{se2packages}

\usepackage[UKenglish]{babel}
\usepackage{hvlogos}

\begin{document}
  \DocInput{\jobname.dtx}
\end{document}
%</driver>
% \fi
%
% \GetFileInfo{se2packages.sty}
%
% \title{^^A
%   \pkg{se2packages}---Supporting Packages for the \pkg{se2thesis} Bundle^^A
%   \thanks{This file describes \fileversion, last revised \filedate.}^^A
% }
%
% \author{^^A
%   Stephan Lukasczyk^^A
%   \thanks{Email: \href{mailto:stephan@dante.de}{stephan@dante.de}}^^A
% }
%
% \date{Released \filedate}
%
% \maketitle
%
% \begin{documentation}
%
% \section{The \pkg{se2packages} package}\label{sec:doc-se2packages}
%
% Several packages can be useful for writing a thesis.
% We provide a simple package for the user
% that loads these support packages with the recommended options.
% Please note that you might not need all these packages,
% however,
% having a look at them (especially their documentation)
% might give you an insight whether to use a certain package.
% Our general recommendation is to use as few packages as you can;
% some might have conflicts,
% others basically do the same,
% or are outdated.
% Please consider reading the documentation of each package you are using
% to figure out whether they have any conflicts with other packages
% (for example, one cannot use the recommended \pkg{siunitx} package
% together with \pkg{SIunits}) or they might require to be loaded
% at special places in your preamble (for example, \pkg{hyperref}
% is usually meant to be loaded as the last package,
% except you are also using \pkg{cleveref},
% which needs to be loaded \emph{after} \pkg{hyperref}).
%
% \subsection{The Recommended Packages}
%
% We first present the packages we recommend
% and give a rationale why we believe they should be used.
% Afterwards,
% we will discuss how to load the \pkg{se2packages} package
% to use one or the other of these recommended packages
% in the next section.
%
% \subsubsection{Quoting with \pkg{csquotes}}
%
% The \pkg{csquotes} packages allows you for intelligent quoting of text.
% While verbose quotes are not that common in computer science,
% the package still provides some useful macros to the user.
%
% \subsubsection{Number formatting with \pkg{siunitx}}
%
% While \pkg{siunitx}'s original purpose was to format physical quantities,
% it provides a lot of useful features
% when typesetting theses (and other documents) in computer science.
% When you skim through its documentation,
% especiall look at the \cmd{\qty} and \cmd{\num} macros,
% as well as the section on typesetting tabular material.
% We also recommend reading an extensive discussion on number formatting,
% precision of presented numbers,
% and more related topics in Beyer et al.'s journal paper
% on requirements and solutions for reliable
% benchmarking~\cite{DBLP:journals/sttt/BeyerLW19}.
%
% \subsubsection{Code Listings}
%
% We prefer using the \pkg{minted} package for code listings
% because it provides a lot of options for styling
% as well as built-in support for a huge number of languages.
% However,
% this package requires the installation of Python
% and the setting of the |-shell-escape| option to your \TeX{} engine.
% Please read the package's documentation to set it up;
% the aforementioned Docker images from the Island of \TeX{} project
% provide you everything set up already.
%
% In case you do not want to use the \pkg{minted} package,
% please consider using \pkg{listings} for typesetting your code listings.
%
% Please be aware to use \emph{either} \pkg{minted} \emph{or} \pkg{listings}!
%
% \emph{Attention:} The way how \pkg{minted} is implemented causes the resulting
% code to be typeset in gray scale, if you use the \pkg{se2colors} package with
% option |cmode=bw| or the \cls{se2thesis} class with |colormode=bw| (or their
% respective aliases).  Students were requesting that they want to use the
% black-and-white mode of \pkg{se2color} because it saves lots of money when
% printing their thesis, while they want the colours in the syntax highlighting
% of \pkg{minted}.  One can achieve this by adding hooks to the preamble of the
% document, after loading the \pkg{minted} package (which is recommend by
% setting |minted=true| as an option to \pkg{se2packages}):
% \begin{LaTeXdemo}[code only]
% \AddToHook{env/minted/begin}{\selectcolormodel{cmyk}}
% \AddToHook{cmd/mintinline/before}{\selectcolormodel{cmyk}}
% \end{LaTeXdemo}
% Thanks to Ulrike Fischer for pointing out this hook mechanism.\footnote{%
%   \href{https://tex.stackexchange.com/a/718719/14622}{tex.stackexchange.com/a/718719/14622}
% }
%
% Another option, if you do not want colours in the \pkg{minted} output when you
% use the gray-scale mode of \pkg{se2colors} is the change the highlighting
% style, \pkg{minted} uses.  The package for this provides the macro
% \cs{usemintedstyle}, which accepts a style available in your local
% \texttt{pygments} installation, see the \texttt{pygments}
% documentation\footnote{%
%   \href{https://pygments.org/styles/}{pygments.org/styles/}
% } for an overview over the available styles; the |bw| style is explicitly
% designed for this kind of scenario.
%
% \subsubsection{Designing Tables}
%
% A basically mandatory package to all users of tables
% is the \pkg{booktabs} package.
% Especially its documentation is a must-read!
% It provides a large variety of hints for designing tables,
% most notably that one should never ever use vertical lines;
% horizontal lines should be used sparesly;
% \pkg{booktabs} provides three macros for lines that shall be used
% instead of \cmd{\hline} from standard \LaTeX:
% \cmd{\toprule} for a rule on the top of a table,
% above the column heads;
% \cmd{\midrule} to separate column heads and the content,
% and \cmd{\bottomrule} to mark the bottom of a table.
% We highly recommend that you follow this style when writing your thesis.
%
% Note that captions of tables shall be put \emph{above} the table
% whereas captions of figures shall go \emph{below} the figure.
% The rationale is that a figure should be more or less self explaining
% while a table almost always needs some explanation.
%
% Unfortunately,
% the distances when using a \cmd{\caption} above a table are wrong by default;
% when creating tables,
% consider loading the \pkg{hvfloats} package
% and use its \cmd{\tabcaption} instead of \cmd{\caption} for tables.
% The \pkg{hvfloats} package furthermore provides additional useful macros
% to typeset all kinds of floats.
%
% \subsubsection{Controlling Ligatures, Orphans, and Widows}
%
% When using \LuaTeX{}~(which we recommend anyway),
% we additionally recommend to use the \pkg{selnolig}
% and \pkg{lua-widow-control} packages.
% The former implements a better control over ligatures,
% the latter provides better control over orphans and widows.
%
% For some reason we were not yet able to fix properly,
% \pkg{lua-widow-control} works on some machine but fails on others,
% although they are using the same version of \TeXLive.
% We therefore deactivate \pkg{lua-widow-control} by default.
%
% \subsubsection{Use \pkg{microtype} for Microtypographic Optimisation}
%
% \pkg{microtype} is a game changer when writing documents.
% While standard \LaTeX{} provides reasonably good typography,
% the \pkg{microtype} package brings it to the next level.
% We refer you to the documentation of the package for details.
%
% Additionally,
% we prefer to have a thin space around the m-dash.
% Although we are aware of the discussion
% whether to have such a space in English,
% we think it looks more beautiful and thus recommend it.
%
% \subsubsection{Use \pkg{biblatex} for Bibliographic References}
%
% The standard way of typesetting bibliographic references used to be \BibTeX.
% The original \BibTeX,
% however,
% seems to be very outdated in various ways:
% it originally only supported 7\,bit character sets
% and creating citation styles requires the usage
% of a kind of an archaic language.
% \BibLaTeX{} resolves many of the drawbacks of \BibTeX;
% when combined with the |biber| engine,
% it supports full UTF-8 unicode,
% therefore correct sorting of the references now works out of the box;
% also creating citation styles can now be done using simple \LaTeX{} commands.
%
% \subsubsection{Use \pkg{cleveref} for Internal References}
%
% \LaTeX{} provides an easy-to-use reference mechanism using the \cmd{\label}
% and \cmd{\ref} macros.
% However,
% this requires some manual effort and the text needs to specify
% whether a reference is to a figure, section, or table.
% We often see in drafts artefacts such as \enquote{we discuss our findings in 4};
% but what is \enquote{4} here?
% Is it a section?
% A table?
% A figure?
%
% To avoid such confusion,
% use the \pkg{cleveref} package,
% which automatically infers the type of the reference~(see its documentation
% on how this works).
% The \pkg{cleveref} package furthermore avoids an additional, common mistake:
% between the name of the element and its reference
% one needs to have a non-breaking space that is often forgotten.
% Instead people are using normal spaces that can break at the end of a line
% which looks very ugly.
%
% Please note that,
% in contrast to most other packages,
% \pkg{cleveref} has to be loaded \emph{after} the \pkg{hyperref} package!
%
% \subsection{Loading \pkg{se2packages}}
%
% When you load the \pkg{se2packages} package
% it will automatically load the \pkg{csquotes} and \pkg{booktabs} packages
% because we believe that you definitely should use them.
% Still, it is possible to avoid this by setting the following
% load-time options to the package.
% \begin{function}{csquotes,nocsquotes}
%   \begin{syntax}
%     |csquotes| = \meta{true,false}
%     |nocsquotes|
%   \end{syntax}
%   The argument to |csquotes| can be one of |true| or |false|.
%   The latter does not load \pkg{csquotes}, the former is the default value.
%   By using |nocsquotes| you can also avoid that \pkg{csquote} is loaded.
% \end{function}
%
% \begin{function}{booktabs,nobooktabs}
%   \begin{syntax}
%     |booktabs| = \meta{true,false}
%     |nobooktabs|
%   \end{syntax}
%   The argument to |booktabs| can be one of |true| or |false|.
%   The latter does not load \pkg{booktabs}, the former is the default value.
%   By using |nobooktabs| you can also avoid that \pkg{booktabs} is loaded.
% \end{function}
%
% The \pkg{siunitx} package is helpful to set quantities
% and to automatically format numbers consistently.
% The following load-time option loads it automatically.
% \begin{function}{siunitx}
%   \begin{syntax}
%     |siunitx| = \meta{true,false}
%   \end{syntax}
%   Decides whether the package loads \pkg{siunitx}
%   and sets the options for \pkg{siunitx} to the values we recommend.
%   The default value is |false|, set it to |true| to load \pkg{siunitx}.
% \end{function}
%
% We provide a load-time option to load \pkg{minted}
% and configure it.
% \begin{function}{minted}
%   \begin{syntax}
%     |minted| = \meta{true,false}
%   \end{syntax}
%   Decides whether the package loads \pkg{minted}
%   and sets the options for \pkg{minted} to the values we recommend.
%   The default value is |false|, set it to |true| to load \pkg{minted}.
%   Note that \pkg{minted} requires additional installation effort
%   and the use of the |-shell-escape| option, see its documentation.
% \end{function}
%
% We also provide a load-time option to load \pkg{listings}
% and configure it.
% This option and the |minted| option are mutually exclusive!
% \begin{function}{listings}
%   \begin{syntax}
%     |listings| = \meta{true,false}
%   \end{syntax}
%   Decides whether the package loads \pkg{listings}
%   and sets the options for \pkg{listings} to the values we recommend.
%   The default value is |false|, set it to |true| to load \pkg{listings}.
% \end{function}
%
% We provide a load-time option to load \pkg{selnolig}.
% \begin{function}{selnolig,noselnolig}
%   \begin{syntax}
%     |selnolig| = \meta{true,false}
%     |noselnolig|
%   \end{syntax}
%   The argument to |selnolig| can be one of |true| or |false|.
%   The latter does not load \pkg{selnolig}, the former is the default value.
%   By using |noselnolig| you can also avoid that \pkg{selnolig} is loaded.
% \end{function}
%
% We provide a load-time option to load \pkg{lua-widow-control}.
% \begin{function}{widowcontrol,nowidowcontrol}
%   \begin{syntax}
%     |widowcontrol| = \meta{true,false}
%     |nowidowcontrol|
%   \end{syntax}
%   The argument to |widowcontrol| can be one of |true| or |false|.
%   The former loads \pkg{lua-widow-control},
%   the latter is the default value.
%   By using |nowidowcontrol| you can also avoid
%   that \pkg{lua-widow-control} is loaded.
% \end{function}
%
% We provide a load-time option to load and configure \pkg{microtype}.
% \begin{function}{microtype,nomicrotype}
%   \begin{syntax}
%     |microtype| = \meta{true,false}
%     |nomicrotype|
%   \end{syntax}
%   The argument to |microtype| can be one of |true| or |false|.
%   The latter does not load \pkg{microtype}, the former is the default value.
%   By using |nomicrotype| you can also avoid that \pkg{microtype} is loaded.
% \end{function}
%
% We provide a load-time option to load and configure \pkg{biblatex}.
% \begin{function}{biblatex}
%   \begin{syntax}
%     |biblatex| = \meta{true,false,alphabetic,numeric}
%   \end{syntax}
%   Decides whether the package loads \pkg{biblatex}
%   and sets the options for \pkg{biblatex} to the values we recommend.
%   The default value is |false|, set it to |true| to load \pkg{biblatex}.
%
%   By setting the value to |alphabetic|, \pkg{se2packages} sets the citation
%   style for \pkg{biblatex} to its alphabetic one; selecting |numeric| loads
%   the numeric style as provided by the \pkg{acmart} bundle of the ACM.  The
%   value |true| also loads the latter style as a default.
%
%   Note that for the |numeric| style, we provide an enhanced \cs{autocite}
%   command, which does not only print the citation mark in the text but also
%   a short entry listing of the bibliography entry in the margin column of the
%   page.  This entry is only printed for the first occurence of a citation per
%   chapter to not clutter the margin column too much.  The other citation
%   commands from \pkg{biblatex} work as usual.
% \end{function}
%
% Additionally,
% we provide a load-time option to load \pkg{cleveref}.
% This option makes use of \LaTeX's hook system to load it as the very last
% thing in the preamble.
% You shall not use this option, but load \pkg{cleveref} on your own,
% if your document requires packages that need to be loaded even \emph{after}
% \pkg{cleveref}.
% \begin{function}{cleveref}
%   \begin{syntax}
%     |cleveref| = \meta{true,false}
%   \end{syntax}
%   Decides whether the package load \pkg{cleveref}.
%   The default value is |false|, set it to |true| to load \pkg{cleveref}.
% \end{function}
%
% Lastly,
% we provide one option to load all recommended packages~(with \pkg{minted}
% prefered over \pkg{listings}).
% \begin{function}{all}
%   \begin{syntax}
%     |all| = \meta{true,false}
%   \end{syntax}
%   Decides whether the package loads all recommended packages.
%   The default value is |false|, set it to |true| to load them.
% \end{function}
%
% \subsection{Additional Recommended Packages}
%
% The following packages are not part of \pkg{se2packages}
% because they serve a very specific purpose.
% They might not be needed by many thesis writers,
% however,
% we mention them here in order to give you a pointer.
% We list them in alphabetic order.
%
% \begin{itemize}
%   \item \pkg{algorithm2e}: provides tooling for typesetting algorithm
%     diagrams.
%     We recommend loading it with the following options:
%     \begin{LaTeXdemo}[code only]
% \usepackage[ruled,noline,noend,linesnumbered]{algorithm2e}
%     \end{LaTeXdemo}
%   \item \pkg{hvfloat}: a package to control captions, fullpage, and
%     double-page floats.
%     We recommend to load it and use its \cmd{\tabcaption} for table captions.
%   \item \pkg{lettrine}: allows to start a paragraph with a large initial.
%     This is a beautiful opener for the first paragraph of a thesis' main
%     content.
%   \item \pkg{mathtools}: an enhanced version of the famous \amsmath
%     package that provides more functionality and fixes some of the known bugs.
%   \item \pkg{pdfx}: adds support for PDF/X and PDF/A, which is useful for
%     creating documents that can be archived, for example, by the University's
%     library.
%   \item \pkg{pgfplots}: a library for creating all kinds of plots, based on
%     \pkg{pgf}, which is also the basis of the \pkg{tikz} package.
%   \item \pkg{sidenotesplus}: a package to place all kinds of objects in the
%     margin of a document, from footnotes (then called sidenotes) to figures
%     and tables.
%   \item \pkg{subcaption}: support for sub-captions and sub-figures.
%   \item \pkg{tabularray}: a rewrite of \LaTeX's array and tabular abilities
%     in \LaTeXIII.
%     While the package offers a great combination of several other packages'
%     functionality, it tends to be quite slow for large tables due to the way
%     it is implemented internally.
%   \item \pkg{tcolorbox}: allows to typeset coloured boxes, for example, for
%     short summaries on research questions or hypotheses.
%   \item \pkg{tikz}: for creating all kinds of figures.
%     Beware: \TikZ{} is a really huge package with a lot of functionality and
%     a manual of more than 1300~pages.
% \end{itemize}
%
%
% \begin{thebibliography}{9}
%   \bibitem{DBLP:journals/sttt/BeyerLW19} Dirk Beyer, Stefan Löwe, and Philipp
%   Wendler: \emph{Reliable benchmarking: requirements and solutions}. STTT
%   21(1): 1--29 (2019)
% \end{thebibliography}
% 
%
%
%
% \end{documentation}
%
% \begin{implementation}
%
% \section{\pkg{se2packages} implementation}
%
% Start the \pkg{DocStrip} guards.
%    \begin{macrocode}
%<*package>
%    \end{macrocode}
%
% Identify the internal prefix (\LaTeX3 \pkg{DocStrip} convention): only
% internal material in this \emph{submodule} should be used directly.
%    \begin{macrocode}
%<@@=slcd_packages>
%    \end{macrocode}
%
% Identify the package and give the overall version information.
%    \begin{macrocode}
\ProvidesExplPackage {se2packages} {2024-06-25} {4.2.0}
  {Supporting packages for the se2thesis bundle}
%    \end{macrocode}
%
% \subsection{Load-time options}
%
% Before we define the actual options,
% define some Boolean variables to store the selected values.
%
% \begin{macro}{\l_@@_csquotes_bool}
%   Whether or not to load \pkg{csquotes}.
%    \begin{macrocode}
\bool_new:N \l_@@_csquotes_bool
%    \end{macrocode}
% \end{macro}
%
% \begin{macro}{\l_@@_booktabs_bool}
%   Whether or not to load \pkg{booktabs}.
%    \begin{macrocode}
\bool_new:N \l_@@_booktabs_bool
%    \end{macrocode}
% \end{macro}
%
% \begin{macro}{\l_@@_siunitx_bool}
%   Whether or not to load and configure \pkg{siunitx}.
%    \begin{macrocode}
\bool_new:N \l_@@_siunitx_bool
%    \end{macrocode}
% \end{macro}
%
% \begin{macro}{\l_@@_minted_bool}
%   Whether or not to load and configure \pkg{minted}.
%    \begin{macrocode}
\bool_new:N \l_@@_minted_bool
%    \end{macrocode}
% \end{macro}
%
% \begin{macro}{\l_@@_listings_bool}
%   Whether or not to load and configure \pkg{listings}.
%    \begin{macrocode}
\bool_new:N \l_@@_listings_bool
%    \end{macrocode}
% \end{macro}
%
% \begin{macro}{\l_@@_biblatex_tl}
%   Whether or not to load and configure \pkg{biblatex}.
%    \begin{macrocode}
\tl_new:N \l_@@_biblatex_tl
%    \end{macrocode}
% \end{macro}
%
% \begin{macro}{\l_@@_selnolig_bool}
%   Whether or not to load \pkg{selnolig} (\LuaTeX{} only).
%    \begin{macrocode}
\bool_new:N \l_@@_selnolig_bool
%    \end{macrocode}
% \end{macro}
%
% \begin{macro}{\l_@@_luawidowcontrol_bool}
%   Whether or not to load \pkg{luawidowcontrol} (\LuaTeX{} only).
%    \begin{macrocode}
\bool_new:N \l_@@_luawidowcontrol_bool
%    \end{macrocode}
% \end{macro}
%
% \begin{macro}{\l_@@_microtype_bool}
%   Whether or not to load \pkg{microtype}.
%    \begin{macrocode}
\bool_new:N \l_@@_microtype_bool
%    \end{macrocode}
% \end{macro}
%
% \begin{macro}{\l_@@_cleveref_bool}
%   Whether or not to load and configure \pkg{cleveref}.
%    \begin{macrocode}
\bool_new:N \l_@@_cleveref_bool
%    \end{macrocode}
% \end{macro}
%
% \begin{macro}{\l_@@_all_bool}
%   Whether or not to load and configure all recommended packages.
%    \begin{macrocode}
\bool_new:N \l_@@_all_bool
%    \end{macrocode}
% \end{macro}
%
% Define the actual load-time options.
%    \begin{macrocode}
\keys_define:nn { seiipackages }
  {
    csquotes .bool_gset:N = \l_@@_csquotes_bool,
    csquotes .initial:n = true,
    nocsquotes .meta:n = {csquotes=false},

    booktabs .bool_gset:N = \l_@@_booktabs_bool,
    booktabs .initial:n = true,
    nobooktabs .meta:n = {booktabs=false},

    siunitx .bool_gset:N = \l_@@_siunitx_bool,
    siunitx .initial:n = false,

    minted .bool_gset:N = \l_@@_minted_bool,
    minted .initial:n = false,

    listings .bool_gset:N = \l_@@_listings_bool,
    listings .initial:n = false,

    biblatex .tl_gset:N = \l_@@_biblatex_tl,
    biblatex .initial:n = false,

    selnolig .bool_gset:N = \l_@@_selnolig_bool,
    selnolig .initial:n = true,
    noselnolig .meta:n = {selnolig=false},

    widowcontrol .bool_gset:N = \l_@@_luawidowcontrol_bool,
    widowcontrol .initial:n = false,
    nowidowcontrol .meta:n = {widowcontrol=false},

    microtype .bool_gset:N = \l_@@_microtype_bool,
    microtype .initial:n = true,
    nomicrotype .meta:n = {microtype=false},

    cleveref .bool_gset:N = \l_@@_cleveref_bool,
    cleveref .initial:n = false,

    all .bool_gset:N = \l_@@_all_bool,
    all .initial:n = false,
  }
%    \end{macrocode}
%
% \subsection{Option handling}
%
%    \begin{macrocode}
\IfFormatAtLeastTF { 2022-06-01 }
  { \ProcessKeyOptions [ seiipackages ] }
  {
    \RequirePackage{ l3keys2e }
    \ProcessKeysOptions { seiipackages }
  }
%    \end{macrocode}
%
% Define an error message if both \pkg{minted} and \pkg{listings} are requested.
%    \begin{macrocode}
\msg_set:nnnn { seiipackages } { conflicting-packages }
  { Setting~ both~ minted~ and~ listings~ to~ true~ is~ not~ possible. }
  { Choose~ either~ of~ them! }
%    \end{macrocode}
%
% Define a message for logging that a specific package will be loaded by
% \pkg{se2packages}.
% Expects as parameter the name of the package that will be loaded.
%    \begin{macrocode}
\msg_new:nnn { seiipackages } { load-package }
  { se2packages~ loads~ the~ #1~ package now. }
%    \end{macrocode}
%
% And, in case both values are set to |true|,
% raise the error message.
%    \begin{macrocode}
\bool_lazy_and:nnT { \l_@@_minted_bool } { \l_@@_listings_bool }
  {
    \msg_error:nn { seiipackages } { conflicting-packages }
  }
%    \end{macrocode}
%
% In case the |all| option is set,
% set all other values to |true|.
%    \begin{macrocode}
\bool_if:NT \l_@@_all_bool
  {
    \bool_gset_true:N \l_@@_csquotes_bool
    \bool_gset_true:N \l_@@_booktabs_bool
    \bool_gset_true:N \l_@@_siunitx_bool
    \bool_gset_true:N \l_@@_minted_bool
    \bool_gset_false:N \l_@@_listings_bool
    \bool_gset_true:N \l_@@_selnolig_bool
    \bool_gset_true:N \l_@@_luawidowcontrol_bool
    \bool_gset_true:N \l_@@_microtype_bool
    \tl_gset:Nn \l_@@_biblatex_bool {true}
    \bool_gset_true:N \l_@@_cleveref_bool
  }
%    \end{macrocode}
%
% \subsection{Package loading}
%
% Conditionally load the \pkg{csquotes} package.
%    \begin{macrocode}
\bool_if:NT \l_@@_csquotes_bool
  {
    \msg_info:nnn { seiipackages } { load-package } { csquotes }
    \RequirePackage{fvextra}
    \RequirePackage{csquotes}
  }
%    \end{macrocode}
%
% Conditionally load the \pkg{booktabs} package.
%    \begin{macrocode}
\bool_if:NT \l_@@_booktabs_bool
  {
    \msg_info:nnn { seiipackages } { load-package } { booktabs }
    \RequirePackage{booktabs}
  }
%    \end{macrocode}
%
% Conditionally load and configure the \pkg{siunitx} package.
%    \begin{macrocode}
\bool_if:NT \l_@@_siunitx_bool
  {
    \msg_info:nnn { seiipackages } { load-package } { siunitx }
    \RequirePackage[
      add-integer-zero=false,
      free-standing-units=true,
      group-minimum-digits=4,
      list-final-separator={,~ and~ },
      round-mode=figures,
      round-precision=3,
      separate-uncertainty=true,
      uncertainty-mode=separate,
      mode=match,
    ]{siunitx}
  }
%    \end{macrocode}
%
% Conditionally load and configure the \pkg{minted} package.
%    \begin{macrocode}
\bool_if:NT \l_@@_minted_bool
  {
    \msg_info:nnn { seiipackages } { load-package } { minted }
    \RequirePackage[newfloat=true]{minted}
    \setminted{
      autogobble,
      breaklines=true,
      fontsize=\footnotesize,
      frame=single,
      linenos=false,
      resetmargins=true,
      xleftmargin=1em,
      xrightmargin=1em,
    }
  }
%    \end{macrocode}
%
% Conditionally load and configure the \pkg{listings} package.
%    \begin{macrocode}
\bool_if:NT \l_@@_listings_bool
  {
    \msg_info:nnn { seiipackages } { load-package } { listings }
    \RequirePackage{listings}
    \lstset{
      basicstyle=\footnotesize\ttfamily,
      breaklines=true,
      captionpos=t,
      extendedchars=true,
      frame=single,
      keywordstyle=\color{blue}\bfseries,
      showspaces=false,
      showtabs=false,
      showstringspaces=false,
      tabsize=2,
    }
  }
%    \end{macrocode}
%
% Conditionally load \pkg{selnolig} if present and we run \LuaTeX.
%    \begin{macrocode}
\bool_lazy_and:nnT { \l_@@_selnolig_bool } { \sys_if_engine_luatex_p: }
  {
    \IfFileExists { selnolig.sty }
      {
        \RequirePackage{selnolig}
      } {
        \msg:nnnn { seiipackages }
          { selnolig-not-available }
          { Could~ not~ find~ selnolig.sty }
          { You~ might~ want~ to~ install~ it~ for~ better~ ligatures~ control.}
        \msg_note:nn { seipackages } { selnolig-not-available }
      }
  }
%    \end{macrocode}
%
% Conditionally load \pkg{luawidowcontrol} if present and we run \LuaTeX.
%    \begin{macrocode}
\bool_lazy_and:nnT { \l_@@_luawidowcontrol_bool } { \sys_if_engine_luatex_p: }
  {
    \IfFileExists { lua-widow-control.sty }
      {
        \RequirePackage{lua-widow-control}
      } {
        \msg:nnnn { seiipackages }
          { lua-widow-control-not-available }
          { Could~ not~ find~ lua-widow-control.sty }
          { 
            You~ might~ want~ to~ install~ it~ for~ better~ control~ over~
            widows~ and~ orphans.
          }
        \msg_note:nn { seipackages } { lua-widow-control-not-available }
      }
  }
%    \end{macrocode}
%
% Conditionally load and configure \pkg{microtype}.
% If a package option requires for the \pkg{lua-widow-control} package,
% we do not set penalties for widows and orphans
% but let that package control these penalties.
%    \begin{macrocode}
\bool_if:NT \l_@@_microtype_bool
  {
    \msg_info:nnn { seiipackages } { load-package } { microtype }
    \RequirePackage{microtype}
    \bool_if:NF \l_@@_luawidowcontrol_bool
      {
        \clubpenalty=10000
        \widowpenalty=10000
        \displaywidowpenalty=10000
      }
    \SetExtraKerning{
      encoding = {OT1,T1,T2A,LY1,OT4,QX,T5,TS1,EU1,EU2}
    }{
      \textemdash = {167,167},
      — = {167,167}
    }
  }
%    \end{macrocode}
%
% Conditionally load and configure the \pkg{biblatex} package.
%    \begin{macrocode}
\tl_if_eq:NnT \l_@@_biblatex_tl {alphabetic}
  {
    \msg_info:nnn { seiipackages } { load-package } { biblatex }
    \PassOptionsToPackage
      {
        backend=biber,
        backref=true,
        datamodel=software,
        giveninits=true,
        hyperref=auto,
        maxnames=100,
        minalphanames=3,
        sorting=nyt,
        style=alphabetic,
      } { biblatex }
      \RequirePackage{biblatex}
      \RequirePackage{software-biblatex}
      \ExecuteBibliographyOptions{
        halid=false,
        swhid=true,
        shortswhid=false,
        swlabels=true,
        vcs=true,
        license=true,
      }
  }

\bool_new:N \l_@@_biblatex_numeric_variants_bool
\tl_if_eq:NnT \l_@@_biblatex_tl {true}
  {
    \bool_gset_true:N \l_@@_biblatex_numeric_variants_bool
  }
\tl_if_eq:NnT \l_@@_biblatex_tl {numeric}
  {
    \bool_gset_true:N \l_@@_biblatex_numeric_variants_bool
  }

\bool_if:NT \l_@@_biblatex_numeric_variants_bool
  {
    \msg_info:nnn { seiipackages } { load-package } { biblatex }
    \PassOptionsToPackage
      {
        backend=biber,
        backref=true,
        citereset=chapter+,
        citetracker=context,
        giveninits=true,
        hyperref=auto,
        sorting=nyt,
        datamodel=acmdatamodel,
        style=acmnumeric,
      } { biblatex }
      \RequirePackage{biblatex}
      \RequirePackage{ragged2e}
      \RequirePackage[mark=arabic,shape=up,Ragged]{sidenotesplus}

      \newbibmacro{cite:authoryear}{%
        \printtext[labelnumberwidth]{%
          \usebibmacro{cite}%
        }%
        \setunit{\addspace}%
        \printnames{labelname}%
        \setunit{\labelnamepunct}\newblock%
        \usebibmacro{year}%
        \newunit%
        \printfield[title]{labeltitle}
      }
      
      \newrobustcmd*{\makebibmarginnote}[1]{%
        \sidenote*{\blxmkbibnote{foot}{#1}}}
      
      \DeclareCiteCommand{\maycite}[\makebibmarginnote]
        {\usebibmacro{prenote}}
        {\usebibmacro{citeindex}%
         \usebibmacro{cite:authoryear}}
        {\multicitedelim}
        {\usebibmacro{postnote}}
      
      \newcommand*{\cbx@savedcites}{}
      
      \newcommand*{\cbx@margcitewrap}[1]{%
        \global\let\cbx@savedcites\empty
        \mkbibbrackets{#1}%
        \cbx@savedcites
      }
      
      \DeclareCiteCommand{\margincite}[\cbx@margcitewrap]
        {\usebibmacro{prenote}}
        {\usebibmacro{citeindex}%
         \usebibmacro{cite}%
         \ifciteseen
           {}
           {\xappto\cbx@savedcites{\noexpand\maycite{\thefield{entrykey}}}}}
        {\multicitedelim}
        {\usebibmacro{postnote}}
      
      \DeclareMultiCiteCommand{\margincites}[\cbx@margcitewrap]{\margincite}{\multicitedelim}
      
      \DeclareAutoCiteCommand{margin}{\margincite}{\margincites}
      \ExecuteBibliographyOptions{autocite=margin}
      \newcommand{\bibliofont}{\footnotesize}
      \DeclareFieldFormat{titlecase}{#1}% Preserve capitalisation of titles
  }
%    \end{macrocode}
%
% Conditionally load the \pkg{cleveref} package as the last of the preamble.
%    \begin{macrocode}
\bool_if:NT \l_@@_cleveref_bool
  {
    \msg_info:nnn { seiipackages } { load-package } { cleveref }
    \AddToHook { begindocument/before }
      {
        \RequirePackage[capitalise]{cleveref}
        \Crefname{resq}{Research Question}{Research Questions}
        \Crefname{hyp}{Hypothesis}{Hypotheses}
      }
  }
%    \end{macrocode}
%
%
%
%
%    \begin{macrocode}
%</package>
%    \end{macrocode}
%
% \end{implementation}
%
% \PrintIndex
