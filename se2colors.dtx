% \iffalse meta-comment
%
% File: se2colors.dtx Copyright (C) 2022--2024 Stephan Lukasczyk
%
% It may be distributed and/or modified under the conditions of the
% LaTeX Project Public License (LPPL), either version 1.3c of this
% license or (at your option) any later version.  The latest version
% of this license is in the file
%
%    https://www.latex-project.org/lppl.txt
%
% This file is part of the "se2thesis bundle" (The Work in LPPL)
% and all files in that bundle must be distributed together.
%
% The released version of this bundle is available from CTAN.
%
% ---------------------------------------------------------------------
%
% The development version of the bundle can be found at
%
%    https://github.com/se2p/se2thesis
%
% for those people who are interested.
%
% ---------------------------------------------------------------------
%
%<*driver>
\documentclass{l3doc}
% The next line is needed so that \GetFileInfo will be able to pick up
% version data.
\usepackage{se2colors}

\usepackage[UKenglish]{babel}
\usepackage{hvfloat}
\usepackage{hvlogos}
\usepackage{booktabs}

% Taken from xcolor.dtx
\makeatletter
\def\testclr#1#{\@testclr{#1}}
\def\@testclr#1#2{{\fboxsep\z@\fbox{\colorbox#1{#2}{\phantom{XXX}}}}}
\makeatother

\begin{document}
  \DocInput{\jobname.dtx}
\end{document}
%</driver>
% \fi
%
% \GetFileInfo{se2colors.sty}
%
% \title{^^A
%   \pkg{se2colors}---Colour Definitions for the \pkg{se2thesis} Bundle^^A
%   \thanks{This file describes \fileversion, last revised \filedate.}^^A
% }
%
% \author{^^A
%   Stephan Lukasczyk^^A
%   \thanks{^^A
%     E-mail: \href{mailto:stephan@dante.de}{stephan@dante.de}^^A
%   }^^A
% }
%
% \date{Released \filedate}
%
% \maketitle
%
% \begin{documentation}
%
% \section{The \pkg{se2colors} package}\label{sec:doc-se2colors}
%
% Several colours are specific to the university
% and we want to have a comprehensive interface
% to access them throughout all our packages.
%
% The \pkg{se2colors} package provides this exact features.
% One can load it using |\usepackage{se2colors}| in the document preamble.
%
% \begin{function}[updated = 2022-11-03]{cmode}
%   \begin{syntax}
%     |cmode| = \meta{choice}
%   \end{syntax}
%   Selects the colour mode that shall be used for creating the results,
%   a choice from the options specified in \cref{tab:coloursoptions}.
%   The default setting is |4C|.
% \end{function}
%
% \begin{table}[th]
%   \tabcaption{\label{tab:coloursoptions}%
%     Options provided by \pkg{se2colors}.%
%   }
%   \centering
%   \begin{tabular}{@{} l l @{}} \toprule
%     Option & Description \\ \midrule
%     %
%     |cmode=4C|
%            & Define colours in CMYK colour space (\emph{default}). \\
%     %
%     |cmode=CMYK|
%            & Alias for the previous. \\
%     %
%     |CMYK|, |cmyk|
%            & Aliases for the previous. \\
%     %
%     |cmode=RGB|
%            & Define colours in RGB colour space. \\
%     %
%     |RGB|, |rgb|
%            & Aliases for the previous. \\
%     %
%     |cmode=BW|
%            & Define colours in black-and-white colour space. \\
%     %
%     |cmode=1C|
%            & Alias for the previous. \\
%     %
%     |gray|
%            & Alias for the previous. \\
%     %
%     \bottomrule
%   \end{tabular}
% \end{table}
%
% We define two basic colours that are taken from the University's logo,
% namely |UPSE2-Gray| \testclr{UPSE2-Gray}
% and |UPSE2-Orange| \testclr{UPSE2-Orange}.
%
% Additionally,
% we define four colours for the four faculties of the University:
% \begin{itemize}
%   \item the Faculty of Law |UPSE2-JUR| \testclr{UPSE2-JUR},
%   \item the School of Business, Economics and Information Systems
%     |UPSE2-WIWI| \testclr{UPSE2-WIWI},
%   \item the Faculty of Arts and Humanities
%     |UPSE2-PHIL| \testclr{UPSE2-PHIL}, and
%   \item the Faculty of Computer Science and Mathematics
%     |UPSE2-FIM| \testclr{UPSE2-FIM}.
% \end{itemize}
%
% Finally,
% we define a full set of supplementary colours:
% \begin{itemize}
%   \item |UPSE2-DarkGray| \testclr{UPSE2-DarkGray}
%     with its derivatives \\
%     |UPSE2-DarkGray1| \testclr{UPSE2-DarkGray1},
%     |UPSE2-DarkGray2| \testclr{UPSE2-DarkGray2},
%     |UPSE2-DarkGray3| \testclr{UPSE2-DarkGray3},
%     |UPSE2-DarkGray4| \testclr{UPSE2-DarkGray4}, and
%     |UPSE2-DarkGray5| \testclr{UPSE2-DarkGray5}.
%   \item |UPSE2-White| \testclr{UPSE2-White}
%     with its derivatives \\
%     |UPSE2-White1| \testclr{UPSE2-White1},
%     |UPSE2-White2| \testclr{UPSE2-White2},
%     |UPSE2-White3| \testclr{UPSE2-White3},
%     |UPSE2-White4| \testclr{UPSE2-White4}, and
%     |UPSE2-White5| \testclr{UPSE2-White5}.
%   \item |UPSE2-MediumGray| \testclr{UPSE2-MediumGray}
%     with its derivatives \\
%     |UPSE2-MediumGray1| \testclr{UPSE2-MediumGray1},
%     |UPSE2-MediumGray2| \testclr{UPSE2-MediumGray2},
%     |UPSE2-MediumGray3| \testclr{UPSE2-MediumGray3},
%     |UPSE2-MediumGray4| \testclr{UPSE2-MediumGray4}, and
%     |UPSE2-MediumGray5| \testclr{UPSE2-MediumGray5}.
%   \item |UPSE2-LightGray| \testclr{UPSE2-LightGray}
%     with its derivatives \\
%     |UPSE2-LightGray1| \testclr{UPSE2-LightGray1},
%     |UPSE2-LightGray2| \testclr{UPSE2-LightGray2},
%     |UPSE2-LightGray3| \testclr{UPSE2-LightGray3},
%     |UPSE2-LightGray4| \testclr{UPSE2-LightGray4}, and
%     |UPSE2-LightGray5| \testclr{UPSE2-LightGray5}.
%   \item |UPSE2-Orange| \testclr{UPSE2-Orange}
%     with its derivatives \\
%     |UPSE2-Orange1| \testclr{UPSE2-Orange1},
%     |UPSE2-Orange2| \testclr{UPSE2-Orange2},
%     |UPSE2-Orange3| \testclr{UPSE2-Orange3},
%     |UPSE2-Orange4| \testclr{UPSE2-Orange4}, and
%     |UPSE2-Orange5| \testclr{UPSE2-Orange5}.
%   \item |UPSE2-LightRed| \testclr{UPSE2-LightRed}
%     with its derivatives \\
%     |UPSE2-LightRed1| \testclr{UPSE2-LightRed1},
%     |UPSE2-LightRed2| \testclr{UPSE2-LightRed2},
%     |UPSE2-LightRed3| \testclr{UPSE2-LightRed3},
%     |UPSE2-LightRed4| \testclr{UPSE2-LightRed4}, and
%     |UPSE2-LightRed5| \testclr{UPSE2-LightRed5}.
%   \item |UPSE2-DarkRed| \testclr{UPSE2-DarkRed}
%     with its derivatives \\
%     |UPSE2-DarkRed1| \testclr{UPSE2-DarkRed1},
%     |UPSE2-DarkRed2| \testclr{UPSE2-DarkRed2},
%     |UPSE2-DarkRed3| \testclr{UPSE2-DarkRed3},
%     |UPSE2-DarkRed4| \testclr{UPSE2-DarkRed4}, and
%     |UPSE2-DarkRed5| \testclr{UPSE2-DarkRed5}.
%   \item |UPSE2-Green| \testclr{UPSE2-Green}
%     with its derivatives \\
%     |UPSE2-Green1| \testclr{UPSE2-Green1},
%     |UPSE2-Green2| \testclr{UPSE2-Green2},
%     |UPSE2-Green3| \testclr{UPSE2-Green3},
%     |UPSE2-Green4| \testclr{UPSE2-Green4}, and
%     |UPSE2-Green5| \testclr{UPSE2-Green5}.
%   \item |UPSE2-Blue| \testclr{UPSE2-Blue}
%     with its derivatives \\
%     |UPSE2-Blue1| \testclr{UPSE2-Blue1},
%     |UPSE2-Blue2| \testclr{UPSE2-Blue2},
%     |UPSE2-Blue3| \testclr{UPSE2-Blue3},
%     |UPSE2-Blue4| \testclr{UPSE2-Blue4}, and
%     |UPSE2-Blue5| \testclr{UPSE2-Blue5}.
%   \item |UPSE2-AccentGray| \testclr{UPSE2-AccentGray}
%     with its derivatives \\
%     |UPSE2-AccentGray1| \testclr{UPSE2-AccentGray1},
%     |UPSE2-AccentGray2| \testclr{UPSE2-AccentGray2},
%     |UPSE2-AccentGray3| \testclr{UPSE2-AccentGray3},
%     |UPSE2-AccentGray4| \testclr{UPSE2-AccentGray4}, and
%     |UPSE2-AccentGray5| \testclr{UPSE2-AccentGray5}.
% \end{itemize}
%
% \end{documentation}
%
% \begin{implementation}
%
% \section{\pkg{se2colors} implementation}
%
% Start the \pkg{DocStrip} guards.
%    \begin{macrocode}
%<*package>
%    \end{macrocode}
%
% Identify the internal prefix (\LaTeX3 \pkg{DocStrip} convention): only
% internal material in this \emph{submodule} should be used directly.
%    \begin{macrocode}
%<@@=slcd_colors>
%    \end{macrocode}
%
% Identify the package and give the overall version information.
%    \begin{macrocode}
\ProvidesExplPackage {se2colors} {2024-05-08} {4.0.0}
  {A colour support package for the se2thesis bundle}
%    \end{macrocode}
%
% \subsection{Load-time options}
%
% \begin{macro}{\l_@@_colormode_tl}
%   Holds the colour mode selected by the user as a package load-time option.
%    \begin{macrocode}
\keys_define:nn { seiicolors }
  {
    cmode .choice:,
    cmode / 4C .code:n = {
      \PassOptionsToPackage{cmyk}{xcolor}
      \tl_gset:Nn \l_@@_colormode_tl {4C}
    },
    cmode / RGB .code:n = {
      \PassOptionsToPackage{rgb}{xcolor}
      \tl_gset:Nn \l_@@_colormode_tl {RGB}
    },
    cmode / BW .code:n = {
      \PassOptionsToPackage{gray}{xcolor}
      \tl_gset:Nn \l_@@_colormode_tl {BW}
    },
    cmode / CMYK .meta:n = {cmode=4C},
    cmode / cmyk .meta:n = {cmode=4C},
    cmode / rgb .meta:n = {cmode=RGB},
    cmode / gray .meta:n = {cmode=BW},
    RGB .meta:n = {cmode=RGB},
    rgb .meta:n = {cmode=RGB},
    CMYK .meta:n = {cmode=4C},
    cmyk .meta:n = {cmode=4C},
    gray .meta:n = {cmode=BW},
  }
\keys_set:nn { seiicolors } { cmode = 4C }
%    \end{macrocode}
% \end{macro}
%
% \subsection{Option handling}
%
%    \begin{macrocode}
\IfFormatAtLeastTF { 2022-06-01 }
  { \ProcessKeyOptions [ seiicolors ] }
  {
    \RequirePackage { l3keys2e }
    \ProcessKeysOptions { seiicolors }
  }
%    \end{macrocode}
%
% \subsection{Colour definitions}
%
% Load the \pkg{xcolor} package for colour definitions.
%    \begin{macrocode}
\RequirePackage{xcolor}
%    \end{macrocode}
%
%
% Define the primary colours gray and orange as given by the University of
% Passau's style guides.
%    \begin{macrocode}
\definecolorset[named]{RGB/cmyk}{UPSE2-}{}{%
  Orange,242,148,0/0.00,0.44,0.95,0.01;%
  Gray,153,159,158/0.40,0.28,0.28,0.00;%
  JUR,229,49,56/0.01,0.90,0.81,0.01;%
  WIWI,0,96,57/0.87,0.20,0.87,0.36;%
  PHIL,188,42,51/0.06,0.93,0.81,0.11;%
  FIM,0,90,161/0.93,0.63,0.01,0.02;%
  DarkGray,77,77,77/0.74,0.67,0.62,0.18;%
  White,255,255,255/0.00,0.00,0.00,0.00;%
  MediumGray,127,127,127/0.52,0.43,0.40,0.02;%
  LightGray,229,229,229/0.08,0.06,0.04,0.00;%
  LightRed,229,49,56/0.01,0.90,0.81,0.01;%
  DarkRed,188,42,51/0.06,0.93,0.80,0.11;%
  Green,0,96,57/0.87,0.20,0.86,0.35;%
  Blue,0,90,161/0.93,0.63,0.01,0.02;%
  AccentGray,153,159,158/0.40,0.28,0.28,0.00%
}
%    \end{macrocode}
%
% Define the derived colours.
%    \begin{macrocode}
\colorlet{UPSE2-DarkGray1}{UPSE2-DarkGray!20!white}
\colorlet{UPSE2-DarkGray2}{UPSE2-DarkGray!40!white}
\colorlet{UPSE2-DarkGray3}{UPSE2-DarkGray!60!white}
\colorlet{UPSE2-DarkGray4}{UPSE2-DarkGray!75!black}
\colorlet{UPSE2-DarkGray5}{UPSE2-DarkGray!50!black}
\colorlet{UPSE2-White1}{UPSE2-White!95!black}
\colorlet{UPSE2-White2}{UPSE2-White!85!black}
\colorlet{UPSE2-White3}{UPSE2-White!75!black}
\colorlet{UPSE2-White4}{UPSE2-White!65!black}
\colorlet{UPSE2-White5}{UPSE2-White!50!black}
\colorlet{UPSE2-MediumGray1}{UPSE2-MediumGray!20!white}
\colorlet{UPSE2-MediumGray2}{UPSE2-MediumGray!40!white}
\colorlet{UPSE2-MediumGray3}{UPSE2-MediumGray!60!white}
\colorlet{UPSE2-MediumGray4}{UPSE2-MediumGray!75!black}
\colorlet{UPSE2-MediumGray5}{UPSE2-MediumGray!50!black}
\colorlet{UPSE2-LightGray1}{UPSE2-LightGray!90!black}
\colorlet{UPSE2-LightGray2}{UPSE2-LightGray!75!black}
\colorlet{UPSE2-LightGray3}{UPSE2-LightGray!55!black}
\colorlet{UPSE2-LightGray4}{UPSE2-LightGray!25!black}
\colorlet{UPSE2-LightGray5}{UPSE2-LightGray!10!black}
\colorlet{UPSE2-Orange1}{UPSE2-Orange!20!white}
\colorlet{UPSE2-Orange2}{UPSE2-Orange!40!white}
\colorlet{UPSE2-Orange3}{UPSE2-Orange!60!white}
\colorlet{UPSE2-Orange4}{UPSE2-Orange!75!black}
\colorlet{UPSE2-Orange5}{UPSE2-Orange!50!black}
\colorlet{UPSE2-LightRed1}{UPSE2-LightRed!20!white}
\colorlet{UPSE2-LightRed2}{UPSE2-LightRed!40!white}
\colorlet{UPSE2-LightRed3}{UPSE2-LightRed!60!white}
\colorlet{UPSE2-LightRed4}{UPSE2-LightRed!75!black}
\colorlet{UPSE2-LightRed5}{UPSE2-LightRed!50!black}
\colorlet{UPSE2-DarkRed1}{UPSE2-DarkRed!20!white}
\colorlet{UPSE2-DarkRed2}{UPSE2-DarkRed!40!white}
\colorlet{UPSE2-DarkRed3}{UPSE2-DarkRed!60!white}
\colorlet{UPSE2-DarkRed4}{UPSE2-DarkRed!75!black}
\colorlet{UPSE2-DarkRed5}{UPSE2-DarkRed!50!black}
\colorlet{UPSE2-Green1}{UPSE2-Green!10!white}
\colorlet{UPSE2-Green2}{UPSE2-Green!25!white}
\colorlet{UPSE2-Green3}{UPSE2-Green!50!white}
\colorlet{UPSE2-Green4}{UPSE2-Green!75!white}
\colorlet{UPSE2-Green5}{UPSE2-Green!90!white}
\colorlet{UPSE2-Blue1}{UPSE2-Blue!20!white}
\colorlet{UPSE2-Blue2}{UPSE2-Blue!40!white}
\colorlet{UPSE2-Blue3}{UPSE2-Blue!60!white}
\colorlet{UPSE2-Blue4}{UPSE2-Blue!75!black}
\colorlet{UPSE2-Blue5}{UPSE2-Blue!50!black}
\colorlet{UPSE2-AccentGray1}{UPSE2-AccentGray!20!white}
\colorlet{UPSE2-AccentGray2}{UPSE2-AccentGray!40!white}
\colorlet{UPSE2-AccentGray3}{UPSE2-AccentGray!60!white}
\colorlet{UPSE2-AccentGray4}{UPSE2-AccentGray!75!black}
\colorlet{UPSE2-AccentGray5}{UPSE2-AccentGray!50!black}
%    \end{macrocode}
%
%    \begin{macrocode}
%</package>
%    \end{macrocode}
%
% \end{implementation}
%
% \PrintIndex
