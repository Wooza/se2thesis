% \iffalse meta-comment
%
% File: se2thesis.dtx Copyright (C) 2022--2023 Stephan Lukasczyk
%
% It may be distributed and/or modified under the conditions of the
% LaTeX Project Public License (LPPL), either version 1.3c of this
% license or (at your option) any later version.  The latest version
% of this license is in the file
%
%    https://www.latex-project.org/lppl.txt
%
% This file is part of the "se2thesis bundle" (The Work in LPPL)
% and all files in that bundle must be distributed together.
%
% The released version of this bundle is available from CTAN.
%
% ----------------------------------------------------------------------
%
% The development version of the bundle can be found at
%
%    https://github.com/se2p/se2thesis
%
% for those people who are interested.
%
% ----------------------------------------------------------------------
%
%<*driver>
\documentclass{l3doc}
% The next line is needed so that \GetFileInfo will be able to pick up
% version data.
\usepackage{se2colors}
\begin{document}
  \DocInput{\jobname.dtx}
\end{document}
%</driver>
% \fi
%
% \GetFileInfo{se2colors.sty}
%
% \title{^^A
%   \pkg{se2thesis}---Overall set up^^A
%   \thanks{This file describes \fileversion, last revised \filedate.}^^A
% }
%
% \author{^^A
%   Stephan Lukasczyk^^A
%   \thanks{^^A
%     E-Mail: \href{mailto:stephan@dante.de}{stephan@dante.de}^^A
%   }^^A
% }
%
% \begin{documentation}
%
% \section{Introduction}\label{sec:doc-intro}
%
% The University of Passau does not provide a common thesis template
% to its students.
% For theses, written at the Chair of Software Engineering~II,
% many students chose between two templates that were provided by different
% people from the chair;
% other students chose from the large variety of templates available from the
% internet, causing each thesis looking differently.
%
% The author of this package provided a template,
% which he initially created for his bachelor and master thesis,
% that was recommended and used by many students.
% The implementation of that template, however, was very hacky and required
% some changes over time.
% This lead to the idea of creating a new template from scratch,
% that shall be used by all our students for their various types of theses,
% from bachelor to PhD level.
% The result is the \pkg{se2thesis} bundle.
%
% The bundle itself consists of several \LaTeX{} classes and packages
% that also allow reuse of various parts of it.
% Its main class is the \pkg{se2thesis} document class,
% an extension of the \KOMAScript{} |scrreprt| document class.
% The packages \pkg{se2colors} and \pkg{se2fonts} provide necessary
% colour and font settings for the \pkg{se2thesis} class;
% the \pkg{se2packages} allows to load several recommended packages~(see
% \cref{sec:doc-se2packages}) with our recommended,
% predefined settings.
% They are available as separate packages, however, to allow their reuse for
% other classes, packages, and projects, as well.
%
% \subsection{Installation}\label{sec:doc-intro-install}
%
% The \pkg{se2thesis} bundle is available from \CTAN\footnote{
%   \href{https://ctan.org/pkg/se2thesis}{ctan.org/pkg/se2thesis}
% }.
% It is part of both \MiKTeX{} and \TeXLive.
% You can find its development version from GitHub\footnote{
%   \href{https://github.com/se2p/se2thesis}{github.com/se2p/se2thesis}
% }.
% The easiest way of installing it is through your \TeX{} distribution;
% please check your distribution's documentation on this.
% Note that it might be necessary to update your \TeX{} distribution
% because they might not provide you updates;
% for example, there is a new version of \TeXLive{} every year
% which needs to be installed separately.
% If you have installed an older version of \pkg{se2thesis}
% consider updating it to the latest version available from \CTAN{}
% through your distribution.
%
% For manual installation, we recommend to clone the Git repository
% and run the \texttt{l3build} tool that we also use for building the bundle.
% Running |l3build ctan| builds the source files and compiles the documentation.
% This creates a folder |build| containing these artefacts;
% |build/doc| contains the documentation,
% |build/local| the source files.
% See the documentation of \texttt{l3build} and your \TeX{} distribution
% on how to install them manually.
% Please note that \texttt{l3build} also provides a |l3build install| target
% to install the bundle files to your local texmf tree.
%
% The “Island of \TeX{}” project provides Docker images of \TeXLive\footnote{%
%   \href{https://islandoftex.gitlab.io/community/projects/docker/}{islandoftex.gitlab.io/community/projects/docker/}
% }.
% If you do not want to install/update \TeXLive{} on your system
% you might want to consider their \texttt{latest} image,
% which contains a recent \TeXLive{} 2023 build.
% They build their images weekly,
% thus also recent releases of \pkg{se2thesis} will be part of the image
% soon after their release date.
%
% \subsection{General Macros}\label{sec:doc-intro-macros}
%
% They all have in common one macro, \cs{IfFormatAtLeastTF};
% this macro is part of the latest \LaTeX{} kernel.
% However, not all users might have upgraded their \TeX{} installation
% to a level using a recently-enough kernel version.
% Therefore, every class and package of this bundle will conditionally
% define the following macro:
%
% \begin{function}{\IfFormatAtLeastTF}
%   \begin{syntax}
%     \cmd{\IfFormatAtLeastTF} \marg{version} \marg{then block} \marg{else-block}
%   \end{syntax}
%   Checks whether the used \LaTeX{} format is at least the one from the
%   given date value.
%   The date needs to be specified either in YYYY/MM/DD or in YYYY-MM-DD
%   format.
% \end{function}
%
% \section{License}\label{sec:doc-license}
%
% Permission is granted to copy, distribute, and/or modify this software under
% the terms of the \LaTeX{} Project Public License~(LPPL), version~1.3c or
% later~(\href{https://www.latex-project.org/lppl.txt}{https://www.latex-project.org/lppl.txt}).
% The software has the status \enquote{maintained}.
%
% \section{The \cls{se2thesis} class}\label{sec:doc-se2thesis}
%
% The \cls{se2thesis} class is the central component of this bundle.
% It provides a wide variety of settings, mostly regarding the title page~(see
% \cref{sec:doc-se2thesis-title}) and the type area~(see
% \cref{sec:doc-se2thesis-typearea}).
%
% We aim to keep the \cls{se2thesis} class relatively small, especially
% considering packages that we load.
% Currently, the class itself loads the \pkg{se2colors}~(see
% \cref{sec:doc-se2colors}) and \pkg{se2fonts}~(see \cref{sec:doc-se2fonts})
% packages.
% The following packages and classes are loaded:
% \pkg{expl3}, \pkg{l3keys2e} in case one uses a \LaTeX{} kernel from before
% 2022--06--01, \pkg{graphicx}, \pkg{translations}, \KOMAScript, \pkg{xcolor},
% \pkg{ifthen}, as well as \pkg{fontspec}, \pkg{unicode-math},
% and \pkg{libertinus-otf} if one uses \LuaTeX{};
% for \pdfLaTeX{} we load \pkg{fontenc}, \pkg{libertinus-type1}, and
% \pkg{AnonymousPro} instead of the latter three.
% Furthermore, we load \pkg{microtype};
% when using \LuaTeX{}, we also load \pkg{lua-widow-control} and \pkg{selnolig}.
%
% However, we recommend to use a couple of further packages, together with some
% further options to those package.  We describe these settings in
% \cref{sec:doc-se2packages}.
% Please consider looking at this section when starting to write your document.
%
% Before we start with a detailed documentation of the class, we explain one
% helper macro that can be used to check whether a package has been loaded.
% \begin{function}[pTF, updated = 2022-09-27]{\slcd_package_if_loaded:n}
%   \begin{syntax}
%     |\slcd_package_if_loaded:nTF| \Arg{package name} \Arg{true code} \Arg{false code}
%   \end{syntax}
%   Checks whether a package has been loaded and conditionally executes code.
% \end{function}
%
% \subsection{Load-time options}\label{sec:doc-se2thesis-options}
%
% The \pkg{se2thesis} class defines several load-time options, all of them
% optional, on top of the options provided by the \KOMAScript{} document
% classes.
% \begin{function}{class}
%   \begin{syntax}
%     |class| = \meta{choice}
%   \end{syntax}
%   Set the base document class.
%   Values are \cls{scrreprt} or \cls{scrbook}.
%   \emph{Note:} you cannot use \cls{scrartcl} because it does not provide
%   the required |chapter| sectioning.
%   Default is |scrbook|.
% \end{function}
% \begin{function}{paper}
%   \begin{syntax}
%     |paper| = \meta{choice}
%   \end{syntax}
%   Set the paper format.
%   Possible values are |a4| or |b5|.
%   Default is |a4|.
% \end{function}
%
% \begin{function}{logofile}
%   \begin{syntax}
%     |logofile| = \marg{path-to-file}
%   \end{syntax}
%   Defines the path to the University's logo for the title page.
% \end{function}
%
% \begin{function}{thesistype}
%   \begin{syntax}
%     |thesistype| = \meta{choice}
%   \end{syntax}
%   Defines the type of the thesis.
%   Possible values are:
%   |bachelor| for a bachelor thesis, |bachelorproposal| for a proposal to
%   a bachelor thesis, |master| for a master thesis, |masterproposal| for
%   a proposal to a master thesis, |phdproposal| for a proposal to a PhD thesis,
%   |phd| for a PhD thesis in the natural sciences and |phdeng| for a PhD thesis
%   in engineering.
% \end{function}
%
% \begin{function}{colormode}
%   \begin{syntax}
%     |colormode| = \meta{choice}
%   \end{syntax}
%   Select the color scheme used by the automatically loaded \pkg{se2colors}
%   package, see \cref{sec:doc-se2colors} for a description.
% \end{function}
%
%
% \subsection{The title page}\label{sec:doc-se2thesis-title}
%
% Designing a title package for a thesis can be complicated.
% There might be some requirements that are not obvious to the user, especially
% considering the positioning of elements.
% The University of Passau, for example, requires the logo to be positioned on
% the top right of a page;
% theses—especially PhD theses that shall be published through the University's
% library system—could be rejected from publication by the library until this is
% fixed.
%
% We thus redeclare the standard \cs{maketitle} macro from \KOMAScript{} and
% customise it to our needs.
% \begin{variable}{\@maketitle}
%   \begin{syntax}
%     \cmd{\@maketitle}
%   \end{syntax}
%   We override the definition of the \cmd{\@maketitle} macro for our needs.
% \end{variable}
% In addition to the macros provided by the \KOMAScript{} classes
% for the title-page values (e.g. \cs{author}, \cs{title},
% we provide some further macros that can be used.
% Setting values to these macros is optional in any case,
% if they are not set, the corresponding value is not put to the title page.
%
% \begin{function}{\version}
%   \begin{syntax}
%     \cmd{\version} \marg{version}
%   \end{syntax}
%   Specify the version of the document.  This can, for example, be a |git| hash
%   of the current version.
% \end{function}
%
% \begin{function}{\degreeprogramme}
%   \begin{syntax}
%     \cmd{\degreeprogramme} \marg{programme-name}
%   \end{syntax}
%   Specify the degree programme the thesis is meant to be accepted in.
%   Possible values are, among others, \enquote{Informatik} if you are writing
%   your thesis in German, or \enquote{Computer Science} if you are writing the
%   thesis in English.
% \end{function}
%
% \begin{function}{\matrnumber}
%   \begin{syntax}
%     \cmd{\matrnumber} \marg{matriculation-number}
%   \end{syntax}
%   Specify the matriculation number of the student writing the thesis.
%   This is required for Bachelor and Master theses.
% \end{function}
%
% \begin{function}{\supervisor, \cosupervisor}
%   \begin{syntax}
%     \cmd{\supervisor} \marg{name}
%     \cmd{\cosupervisor} \marg{name}
%   \end{syntax}
%   Specify the name of your supervisor and co-supervisor.
%   Both people usually are professors.
% \end{function}
%
% \begin{function}{\advisor, \coadvisor}
%   \begin{syntax}
%     \cmd{\advisor} \marg{name}
%     \cmd{\coadvisor} \marg{name}
%   \end{syntax}
%   Specify the name of your advisor and co-advisor.
%   Both people usually are PhD students or postdocs.
% \end{function}
%
% \begin{function}{\department, \institute}
%   \begin{syntax}
%     \cmd{\department} \marg{name}
%     \cmd{\institute} \oarg{short-name} \marg{name}
%   \end{syntax}
%   Specify the department and institute.
%   The department is, for example, \enquote{Faculty of Computer Science and
%   Mathematics}, the institute, for example, \enquote{Chair of Software
%   Engineering~II}.
%   If the \cs{department} value is not specify, we use \enquote{Faculty of
%   Computer Science} as the default value for English theses and
%   \enquote{Fakultät für Informatik und Mathematik} as the default value of
%   German theses.
% \end{function}
%
% \begin{function}{\external}
%   \begin{syntax}
%     \cmd{\external} \marg{name}
%   \end{syntax}
%   Specify the name of an external referee.
% \end{function}
%
% \begin{function}{\location}
%   \begin{syntax}
%     \cmd{\location} \marg{name-of-town}
%   \end{syntax}
%   Specify the name of your residence town for the signature field.
% \end{function}
%
% To define the path to the logo graphics we require a different workflow:
% We do not bundle logo graphics with this package due to legal restrictions.
% They can be downloaded from the University's website; please note that the
% website for downloading the logo graphics is only accessible from within the
% University's campus network or a VPN connection.
% To specify the path to the logo graphics, we provide a load-time option to the
% \cls{se2thesis} class called |logofile|~(see \cref{sec:doc-se2thesis-options}).
%
% When printing the thesis in two-side mode—which we recommend—the back of the
% title page again denotes author and title on the bottom.
% \begin{variable}{\@lowertitleback}
%   \begin{syntax}
%     \cmd{\l@lowertitleback}
%   \end{syntax}
%   Override this internal macro of \KOMAScript{} to print this information on
%   the back side of the title page.
% \end{variable}
%
% Additionally, we provide some interal rewritings to standard macros from
% \KOMAScript{} that allow to automatically split authors using the \cmd{\and}
% command.
% \begin{variable}{\author, \@author}
%   \begin{syntax}
%     \cmd{\author} \marg{author}
%   \end{syntax}
%   We rewrite the definitions of \cmd{\author} and \cmd{\@author} to do this
%   splitting automatically.
%   Additionally, this also adds a correctly translated version of \enquote{and}
%   between the author names if required.
% \end{variable}
%
% \subsection{Type-area settings}\label{sec:doc-se2thesis-typearea}
%
% The \cls{se2thesis} class manipulates the type area compared to the default
% settings of the \KOMAScript{} classes.
% Our settings are inspired by the \pkg{classicthesis} package, which itself is
% inspired by the style used by famous statistician Edward Tufte.
% We provide predefined settings for DIN-A4, DIN-A5, and DIN-B5 papers.
% If you need settings for other paper sizes, please open an issue on this
% package's GitHub repository
% (\href{https://github.com/se2p/se2thesis}{https://github.com/se2p/se2thesis})
% and we will happily include those settings in a future release of this bundle.
%
% Additionally, we are setting the page footer in a way that it contains the
% page numbers in the outer margin and the headmarks split from the page numbers
% by a vertical bar.
%
% \subsection{Abstract for the thesis}\label{sec:doc-se2thesis-abstract}
%
% Each thesis shall come with an abstract that summarises its content.
% The abstract should be written in the language the thesis is written in.
% Additionally, there is the requirement to provide a German abstract if the
% thesis is written in a foreign language.
% \begin{function}{\abstract}
%   \begin{syntax}
%     \cmd{\abstract}
%   \end{syntax}
%   We ensure that the \cmd{\abstract} command is defined for all document
%   classes.
% \end{function}
%
% \DescribeEnv{abstract}
% To typeset an abstract, we provide an environment called \env{abstract}.
% The environment takes an optional argument that specifies the language that is
% used in this abstract.
% Setting the abstract's language will cause its title to change to the
% respective language;
% additionally, hyphenation is also changed for that language.
% \begin{verbatim}
% \begin{abstract}[language]
%   Your abstract text.
% \end{abstract}
% \end{verbatim}
%
% \subsection{Acknowledgements}\label{sec:doc-se2thesis-acknowledgements}
%
% \DescribeEnv{acknowledgements}
% We provide the \env{acknowledgements} environment to typeset acknowledgements
% for your thesis.
% Using this environment is optional.
% Usually, bachelor and master thesis do not contain such an acknowledgements
% section, however, there is no general rule to this.
% \begin{verbatim}
% \begin{acknowledgements}[language]
%   Your acknowledgements.
% \end{acknowledgements}
% \end{verbatim}
%
% \subsection{Document structuring}\label{sec:doc-se2thesis-structuring}
%
% A larger work, such as a thesis, is usually structured in three large blocks:
% a frontmatter that provides all the overview, such as abstract, table of
% contents, etc.,
% a mainmatter that contains all the actual content,
% and a backmatter for appendices.
% \cls{se2thesis} ensures that the following macros are defined because they are
% not provided by all \KOMAScript{} classes.
% \begin{function}{\frontmatter, \mainmatter, \backmatter}
%   \begin{syntax}
%     \cmd{\frontmatter}
%     \cmd{\mainmatter}
%     \cmd{\backmatter}
%   \end{syntax}
%   Switches between frontmatter, mainmatter, and backmatter.
%   Most notably, the frontmatter will have roman page numbers, while the other
%   two will have arabic page numbers.
% \end{function}
%
% \subsection{Authorship declaration}\label{sec:doc-se2thesis-authorship}
%
% The University of Passau requires its students to provide an authorship
% declaration as part of their thesis for submission.
% They provide a template form, which would not fit the style of the
% \cls{se2thesis} class.
% Thus, we provide the \cmd{\authorshipDeclaration} macro to typeset such
% a declaration.
% It uses the original~(German) text of the declaration and fills in the values
% that are specified by the \cmd{\author} and \cmd{\location} macros.
% \begin{function}{\authorshipDeclaration}
%   \begin{syntax}
%     \cmd{\authorshipDeclaration}
%   \end{syntax}
%   Print the authorship declaration text.
% \end{function}
% \emph{Please note:} the authorship declaration will always be printed in
% German, no matter what the language of the thesis is.
% This happens due to legal requirements.
% In order to make this work, you have to load the \pkg{babel} or
% \pkg{polyglossia} package in a way that it also supports German hyphenation.
% For example, use
% \begin{verbatim}
% \usepackage[ngerman,main=UKenglish]{babel}
% \end{verbatim}
% for a thesis with \emph{traditional English}\footnote{
%   there is a nice, probably photoshopped, picture of a Steam setup dialogue
%   stating that American English is a \enquote{simplified version} of British
%   English, see
%   \href{https://jakubmarian.com/is-american-english-simplified-and-british-english-traditional/}{https://jakubmarian.com/is-american-english-simplified-and-british-english-traditional/}.
% } as its main language and support for German.
%
% \begin{function}{\signatureBox}
%   \begin{syntax}
%     \cmd{\signatureBox} \oarg{width} \marg{signature-name}
%   \end{syntax}
%   A helper macro to print the signature box for the authorship declaration.
%   The optional argument \oarg{width} allows to specify a custom width for the
%   signature line.
%   The default is 5\,cm.
%   The mandatory argument \marg{signature-name} specifies the name of the
%   signee, which will be typeset below the signature line.
% \end{function}
%
% \subsection{Research Questions and findings summaries}\label{sec:doc-se2thesis-summaries}
%
% Most theses written at our Chair will require the student to provide some
% empirical evaluation of their work to shed insights whether their proposed
% ideas are actually useful.
% For an empirical study, one needs to specify research questions and maybe also
% hypotheses.
% The \cls{se2thesis} class supports this by providing environments for this.
%
% \DescribeEnv{resq}
% The \env{resq} environment shall be used to specify a research question.
%
% \DescribeEnv{hyp}
% The \env{hyp} environment shall be used to specify a hypothesis.
%
% \DescribeEnv{summary}
% After describing the results, we recommend to give an explicit summary of the
% findings for a research question or hypothesis.
% This summary shall be given in one or two sentences.
% The \env{summary} environment provides a convenient way for this;
% it will be typeset in a highlighted box that is easy to spot and also allows
% readers of the work to quickly grasp the main findings.
% \begin{verbatim}
% \begin{summary}{label-reference}
%   The summary text itself.
% \end{summary}
% \end{verbatim}
% The environment expects as a parameter a label, for example, to a research
% question;
% however, this can also be arbitrary text.
%
% \end{documentation}
%
% \clearpage
%
% \begin{implementation}
%
% \section{Global helpers}\label{sec:impl-global}
%
% These helpers might be useful for many exported packages and classes,
% thus we keep them on the global level of this implementation.
%
%    \begin{macrocode}
%<*init>
%    \end{macrocode}
%
% Load only the essential support (\pkg{expl3}) \enquote{up-front}, and only
% if required.
%    \begin{macrocode}
\@ifundefined{ExplLoaderFileDate}
  { \RequirePackage{expl3} }
  {}
%    \end{macrocode}
%
% Make sure that the version of \pkg{l3kernel} in use is sufficiently new.
% We use \cs{ExplFileDate} as \cs{@ifpackagelater} does not work for
% pre-loaded \pkg{expl3} in the absence of the package.
%    \begin{macrocode}
\@ifl@t@r\ExplLoaderFileDate{2020-01-09}
  {}
  {%
    \PackageError{se2colors}{Support package expl3 too old}
    {%
      You need to update your installation of the bundles 'l3kernel' and
      'l3packages'.\MessageBreak
      Loading~se2colors~will~abort!%
    }%
    \endinput
  }%
%    \end{macrocode}
%
% \begin{macro}{\IfFormatAtLeastTF}
%   This macro is not present in older kernels, thus we use the \LaTeXe{}
%   mechanism as this is correct for this case.
%    \begin{macrocode}
\providecommand \IfFormatAtLeastTF { \@ifl@t@r \fmtversion }
%    \end{macrocode}
% \end{macro}
%
%    \begin{macrocode}
%</init>
%    \end{macrocode}
%
% \section{The \cls{se2thesis} implementation}\label{sec:impl-se2thesis}
%
% Start the \pkg{DocStrip} guards.
%    \begin{macrocode}
%<*class>
%    \end{macrocode}
%
% Identify the internal prefix (\LaTeX3 \pkg{DocStrip} convention): only
% internal material in this \emph{submodule} should be used directly.
%    \begin{macrocode}
%<@@=slcd>
%    \end{macrocode}
%
% Identify the class and give the overall version number.
%    \begin{macrocode}
\ProvidesExplClass {se2thesis} {2024-02-23} {3.3.1}
  {A thesis class for the Chair of Software Engineering II}
%    \end{macrocode}
%
% \begin{macro}
%   {
%     \slcd_package_if_loaded:nTF,
%     \slcd_package_if_loaded:nT,
%     \slcd_package_if_loaded:nF,
%     \slcd_package_if_loaded_p:n
%   }
%   Defines various \LaTeX3{} variants for the internal \cmd{\@ifpackageloaded}
%   macro to check whether a package has been loaded.
%    \begin{macrocode}
\prg_new_conditional:Nnn \slcd_package_if_loaded:n { p, T, F, TF }
  {
    \use:c { @ifpackageloaded }
      { #1 }
      { \prg_return_true: }
      { \prg_return_false: }
  }
%    \end{macrocode}
% \end{macro}
%
% Load required packages early.
%    \begin{macrocode}
\RequirePackage{graphicx}
\RequirePackage{translations}
\LoadDictionary{se2translations}
\DeclareTranslationFallback{version-of-date}{%
  Version~\l_@@_version_tl\ of~\@date
}
\DeclareTranslation{German}{version-of-date}{%
  Version~\l_@@_version_tl\ vom~\@date
}
\DeclareTranslation{English}{version-of-date}{%
  Version~\l_@@_version_tl\ of~\@date
}
%    \end{macrocode}
%
% \subsection{Define Variables}
%
% The following variables are necessary for the argument handling.
% \begin{macro}{\l_@@_paper_int}
%   A variable to store the key of the page size selected by the user.
%    \begin{macrocode}
\int_new:N \l_@@_paper_int
%    \end{macrocode}
% \end{macro}
%
% We also need properties to store class options that are not for us,
% thus shall be handled by the underlying base class.
% \begin{macro}
%   {
%     \l_@@_base_class_tl,
%     \l_@@_clsopts_prop,
%     \l_@@_unknown_clsopts_prop
%   }
%   Store the known and the unknown class options.
%   The latter will be forwarded to the base class later.
%    \begin{macrocode}
\tl_new:N \l_@@_base_class_tl
\prop_new:N \l_@@_clsopts_prop
\prop_new:N \l_@@_unknown_clsopts_prop
%    \end{macrocode}
% \end{macro}
%
% Define internal variables to hold the values of the fields of the title
% page.
% \begin{macro}
%   {
%     \l_@@_version_tl,
%     \l_@@_degreeprogramme_tl,
%     \l_@@_matrnumber_tl,
%     \l_@@_supervisor_tl,
%     \l_@@_cosupervisor_tl,
%     \l_@@_advisor_tl,
%     \l_@@_coadvisor_tl,
%     \l_@@_department_tl,
%     \l_@@_institute_tl,
%     \l_@@_external_tl,
%     \l_@@_logofile_tl,
%     \l_@@_signature_tl,
%     \l_@@_location_tl
%   }
%    \begin{macrocode}
\tl_new:N \l_@@_version_tl
\tl_new:N \l_@@_degreeprogramme_tl
\tl_new:N \l_@@_matrnumber_tl
\tl_new:N \l_@@_supervisor_tl
\tl_new:N \l_@@_cosupervisor_tl
\tl_new:N \l_@@_advisor_tl
\tl_new:N \l_@@_coadvisor_tl
\tl_new:N \l_@@_department_tl
\tl_new:N \l_@@_institute_tl
\tl_new:N \l_@@_external_tl
\tl_new:N \l_@@_logofile_tl
\tl_new:N \l_@@_signature_tl
\tl_new:N \l_@@_location_tl
%    \end{macrocode}
% \end{macro}
%
% Define several dimensions for the \pkg{typearea} package to define the
% package style.
% \begin{macro}
%   {
%     \l_@@_marginspace_dim,
%     \l_@@_headmarkspace_dim,
%     \l_@@_rulespace_dim,
%     \l_@@_pagemark_minipage_dim,
%     \l_@@_ruleraise_dim,
%     \l_@@_rulewidth_dim,
%     \l_@@_rulethickness_dim
%   }
%    \begin{macrocode}
\dim_new:N \l_@@_marginspace_dim
\dim_new:N \l_@@_headmarkspace_dim
\dim_new:N \l_@@_rulespace_dim
\dim_new:N \l_@@_pagemark_minipage_dim
\dim_new:N \l_@@_ruleraise_dim
\dim_new:N \l_@@_rulewidth_dim
\dim_new:N \l_@@_rulethickness_dim
\dim_gset:Nn \l_@@_marginspace_dim { -1.85cm }
\dim_gset:Nn \l_@@_headmarkspace_dim { 0.75cm }
\dim_gset:Nn \l_@@_rulespace_dim { 10pt }
\dim_gset:Nn \l_@@_pagemark_minipage_dim { 1.5cm }
\dim_gset:Nn \l_@@_ruleraise_dim { -5pt }
\dim_gset:Nn \l_@@_rulewidth_dim { 1.25pt }
\dim_gset:Nn \l_@@_rulethickness_dim { 15pt }
%    \end{macrocode}
% \end{macro}
%
% \subsection{Load-time options}
%
% We define the key-value interface for the class.
% \begin{macro}
%   {
%     \l_@@_base_class_tl,
%     \l_@@_unknown_clsopts_prop,
%     \l_@@_paper_int,
%     \l_@@_thesis_type_tl,
%     \l_@@_licensetype_tl,
%     \l_@@_licensemodifier_tl,
%     \l_@@_licenseversion_tl,
%     \l_@@_colormode_tl,
%   }
%    \begin{macrocode}
\keys_define:nn { seiithesis }
  {
    class .choice:,
    class / report .meta:n = {class=scrreprt},
    class / scrreprt .code:n = \tl_gset:Nn \l_@@_base_class_tl {scrreprt},
    class / book .meta:n = {class=scrbook},
    class / scrbook .code:n = \tl_gset:Nn \l_@@_base_class_tl {scrbook},
    class .initial:n = scrbook,

    paper .choices:nn = {a4,b5}{
      \int_gset_eq:NN \l_@@_paper_int \l_keys_choice_int
    },
    paper .initial:n = a4,

    logofile .tl_gset:N = \l_@@_logofile_tl,
    logofile .initial:n =,

    thesistype .choice:,
    thesistype / bachelor .code:n = \tl_gset:Nn \l_@@_thesis_type_tl {bachelor},
    thesistype / bachelorproposal .code:n = {
      \tl_gset:Nn \l_@@_thesis_type_tl {bachelorproposal}
    },
    thesistype / master .code:n = \tl_gset:Nn \l_@@_thesis_type_tl {master},
    thesistype / masterproposal .code:n = {
      \tl_gset:Nn \l_@@_thesis_type_tl {masterproposal}
    },
    thesistype / phd .code:n = \tl_gset:Nn \l_@@_thesis_type_tl {phd},
    thesistype / phdeng .code:n = \tl_gset:Nn \l_@@_thesis_type_tl {phdeng},
    thesistype / phdproposal .code:n = {
      \tl_gset:Nn \l_@@_thesis_type_tl {phdproposal}
    },
    thesistype .initial:n = master,

    colormode .choices:nn = {cmyk,rgb,bw}{
      \tl_gset_eq:NN \l_@@_colormode_tl \l_keys_choice_tl
    },
    colormode .initial:n = cmyk,

    unknown .code:n = {
      \prop_gput:NVn \l_@@_unknown_clsopts_prop \l_keys_key_tl {#1}
    },
  }
%    \end{macrocode}
% \end{macro}
%
% Handle the options
%    \begin{macrocode}
\IfFormatAtLeastTF { 2022-06-01 }
  { \ProcessKeyOptions [ seiithesis ] }
  {
    \RequirePackage{ l3keys2e }
    \ProcessKeysOptions { seiithesis }
  }
%    \end{macrocode}
%
% Handle the known options for base class
%    \begin{macrocode}
\prop_map_inline:Nn \l_@@_clsopts_prop
  {
    \tl_if_empty:nTF {#2}
      { \PassOptionsToClass {#1} {\l_@@_base_class_tl} }
      {
        \clist_map_inline:nn {#2}
          { \PassOptionsToClass {#1=##1} {\l_@@_base_class_tl} }
      }
  }
%    \end{macrocode}
%
% Load the base class
%    \begin{macrocode}
\LoadClass{\l_@@_base_class_tl}
%    \end{macrocode}
%
% Attempt to handle the unknown options
%    \begin{macrocode}
\prop_map_inline:Nn \l_@@_unknown_clsopts_prop
  {
    \cs_if_exist:cT {KV@KOMA.\l_@@_base_class_tl.cls@#1}
      {
        \tl_if_empty:nTF {#2}
          { \KOMAoptions{#1} }
          { \KOMAoption{#1}{#2}}
      }
  }
%    \end{macrocode}
%
% \subsection{Package loading}
%
% We load some packages with options that depend on options to the
% \cls{se2thesis} class.
% Thus, we load them here to be able to hand them over the respective values.
%
% We start with the \pkg{se2colors} and \pkg{se2fonts} packages.
%    \begin{macrocode}
\PassOptionsToPackage{\l_@@_colormode_tl}{se2colors}
\RequirePackage{se2colors}

\RequirePackage{se2fonts}
%    \end{macrocode}
%
% \subsection{User macros for the title page}
%
% In addition to the macros provided by the \KOMAScript{} classes
% for the title-page values (e.g. \cs{author}, \cs{title}),
% provide these additional macros to the user.
% \begin{macro}{\version}
%   Specify the version of the document, e.g., a |git| hash.
%    \begin{macrocode}
\ProvideDocumentCommand \version { m }
  {
    \tl_set:Nn \l_@@_version_tl {#1}
  }
%    \end{macrocode}
% \end{macro}
%
% \begin{macro}{\degreeprogramme}
%   Specify the degree programme the thesis is meant to be accepted in.
%    \begin{macrocode}
\ProvideDocumentCommand \degreeprogramme { m }
  {
    \tl_set:Nn \l_@@_degreeprogramme_tl {#1}
  }
%    \end{macrocode}
% \end{macro}
%
% \begin{macro}{\matrnumber}
%   Specify the matriculation number of the student.
%    \begin{macrocode}
\ProvideDocumentCommand \matrnumber { m }
  {
    \tl_set:Nn \l_@@_matrnumber_tl {#1}
  }
%    \end{macrocode}
% \end{macro}
%
% \begin{macro}{\supervisor, \cosupervisor}
%   Specify the supervisor and co-supervisor of the thesis, usually a professor.
%    \begin{macrocode}
\ProvideDocumentCommand \supervisor { m }
  {
    \tl_set:Nn \l_@@_supervisor_tl {#1}
  }
\ProvideDocumentCommand \cosupervisor { m }
  {
    \tl_set:Nn \l_@@_cosupervisor_tl {#1}
  }
%    \end{macrocode}
% \end{macro}
%
% \begin{macro}{\advisor, \coadvisor}
%   Specify the advisor and co-advisor of the thesis, usually a PhD student or
%   postdoc.
%    \begin{macrocode}
\ProvideDocumentCommand \advisor { m }
  {
    \tl_set:Nn \l_@@_advisor_tl {#1}
  }
\ProvideDocumentCommand \coadvisor { m }
  {
    \tl_set:Nn \l_@@_coadvisor_tl {#1}
  }
%    \end{macrocode}
% \end{macro}
%
% \begin{macro}{\department, \institute}
%   Specify the university's department and institute you are writing
%   the thesis for.
%    \begin{macrocode}
\ProvideDocumentCommand \department { m }
  {
    \tl_set:Nn \l_@@_department_tl {#1}
  }
\ProvideDocumentCommand \institute { o m }
  {
    \tl_set:Nn \l_@@_institute_tl {#2}
  }
%    \end{macrocode}
% \end{macro}
%
% \begin{macro}{\external}
%   Specify an external referee.
%    \begin{macrocode}
\ProvideDocumentCommand \external { m }
  {
    \tl_set:Nn \l_@@_external_tl {#1}
  }
%    \end{macrocode}
% \end{macro}
%
% \begin{macro}{\location}
%   Specify the location for the signature field.
%    \begin{macrocode}
\ProvideDocumentCommand \location { m }
  {
    \tl_set:Nn \l_@@_location_tl {#1}
  }
%    \end{macrocode}
% \end{macro}
%
% \subsection{Define logo, paper size, and paper style}
%
% For the logo on the titlepage, we define further variables to store its height
% and a box to store the logo itself.
% \begin{macro}{\l_@@_logo_height_dim, \l_@@_logo_box}
%    \begin{macrocode}
\dim_if_exist:NF \l_@@_logo_height_dim
  {
    \dim_new:N \l_@@_logo_height_dim
    \int_compare:nTF { \l_@@_paper_int=1 }
      {
        \dim_gset:Nn \l_@@_logo_height_dim { 67.5pt }
      } {
        \dim_gset:Nn \l_@@_logo_height_dim { 56.8pt }
      }
  }
\box_if_exist:NF \l_@@_logo_box
  {
    \box_new:N \l_@@_logo_box
  }
\tl_if_empty:NF \l_@@_logofile_tl
  {
    \hbox_gset:Nn \l_@@_logo_box
      {
        \includegraphics[%
          height=\l_@@_logo_height_dim%
        ]{\l_@@_logofile_tl}
      }
  }
%    \end{macrocode}
% \end{macro}
%
% Set the paper size depending on the selected |paper| option.
%    \begin{macrocode}
\int_compare:nTF { \l_@@_paper_int=1 }
  {
    \areaset[current]{350pt}{567pt}
    \setlength{\marginparsep}{8.5cm}
    \setlength{\marginparsep}{1em}
  }{
    \areaset[current]{350pt}{567pt}
    \setlength{\marginparsep}{7.4cm}
    \setlength{\marginparsep}{0.87em}
  }
%    \end{macrocode}
%
% Provide the package style.
% We start by loading the \pkg{scrlayer-scrpage} package with the appropriate
% options and set some basic properties.
%    \begin{macrocode}
\PassOptionsToPackage{automark}{scrlayer-scrpage}
\RequirePackage{scrlayer-scrpage}
\clearpairofpagestyles
\setkomafont{pagefoot}{\normalfont\sffamily}
%    \end{macrocode}
%
% We can then define the footer for odd pages, which will appear on the right
% side of the page's footer.
% This definition contains first, as an optional argument, the style of a page
% with |pagestyle| set to |empty|, i.e., a page where a new chapter starts, and
% afterwards the style of a regular right-hand side page.
%    \begin{macrocode}
\rofoot[{%
  \group_begin: \ \group_end:
  \footnotesize%
  \hspace*{\l_@@_headmarkspace_dim}%
  \group_begin:
    \color{UPSE2-Blue}%
    \rule[\l_@@_ruleraise_dim]{\l_@@_rulewidth_dim}{\l_@@_rulethickness_dim}%
  \group_end:
  \hspace*{\l_@@_rulespace_dim}%
  \begin{minipage}[b]{\l_@@_pagemark_minipage_dim}%
    \normalsize\textbf{\pagemark}%
  \end{minipage}%
  \hspace{\l_@@_marginspace_dim}%
}]{%
  \group_begin: \ \group_end:
  \footnotesize%
  \group_begin:
    \color{UPSE2-Blue}\headmark
  \group_end:
  \hspace*{\l_@@_rulespace_dim}%
  \group_begin:
    \color{UPSE2-Blue}%
    \rule[\l_@@_ruleraise_dim]{\l_@@_rulewidth_dim}{\l_@@_rulethickness_dim}%
  \group_end:
  \hspace*{\l_@@_rulespace_dim}%
  \begin{minipage}[b]{\l_@@_pagemark_minipage_dim}%
    \normalsize\textbf{\pagemark}%
  \end{minipage}%
  \hspace{\l_@@_marginspace_dim}%
}
%    \end{macrocode}
%
% Similarly, we define the footer for even pages, which will appear on the left
% side of the page's footer.
%    \begin{macrocode}
\lefoot[{%
  \null\hspace{\l_@@_marginspace_dim}%
  \footnotesize%
  \begin{minipage}[b]{\l_@@_pagemark_minipage_dim}%
    \raggedleft\normalsize\textbf{\pagemark}%
  \end{minipage}%
  \footnotesize%
  \hspace*{\l_@@_rulespace_dim}%
  \group_begin:
    \color{UPSE2-Blue}%
    \rule[\l_@@_ruleraise_dim]{\l_@@_rulewidth_dim}{\l_@@_rulethickness_dim}%
  \group_end:
}]{%
  \null\hspace{\l_@@_marginspace_dim}%
  \footnotesize%
  \begin{minipage}[b]{\l_@@_pagemark_minipage_dim}%
    \raggedleft\normalsize\textbf{\pagemark}%
  \end{minipage}%
  \footnotesize%
  \hspace*{\l_@@_rulespace_dim}%
  \group_begin:
    \color{UPSE2-Blue}%
    \rule[\l_@@_ruleraise_dim]{\l_@@_rulewidth_dim}{\l_@@_rulethickness_dim}%
  \group_end:
  \hspace*{\l_@@_headmarkspace_dim}%
  \group_begin:
    \color{UPSE2-Blue}\headmark
  \group_end:
}
%    \end{macrocode}
%
% Finally, set the page style.
%    \begin{macrocode}
\pagestyle{scrheadings}
%    \end{macrocode}
%
% \subsection{The title page}
%
% We start out by adjusting some \KOMAScript{} fonts.
%    \begin{macrocode}
\setkomafont{title}{\Huge}
\setkomafont{subtitle}{\Large}
\setkomafont{subject}{\normalsize}
\setkomafont{author}{\large}
\setkomafont{date}{\normalsize}
\setkomafont{publishers}{\normalsize}
%    \end{macrocode}
%
% \begin{macro}{\author, \@author}
%   Allow for automated splitting of author's names.
%    \begin{macrocode}
\seq_new:N \l_@@_author_seq
\renewcommand*\author[2][]{
  \seq_gset_split:Nnn \l_@@_author_seq {\and} {#2}
  \tl_if_empty:nTF {#1}
    { \tl_set:Nn \l_@@_signature_tl {#2} }
    { \tl_set:Nn \l_@@_signature_tl {#1} }
}
\renewcommand*{\@author}{
  \group_begin:
    \hyphenpenalty=100000
    \seq_use:Nnnn \l_@@_author_seq {~\GetTranslation{and}~} {,~} {~\&~}
  \group_end:
}
%    \end{macrocode}
% \end{macro}
%
% Define a new layer using the functionality from \pkg{scrlayer-scrpage} for the
% logo image.
%    \begin{macrocode}
\DeclareNewLayer[
  mode=picture,
  foreground,
  align=tr,
  hoffset=\oddsidemargin+1.5in+\textwidth,
  voffset=\coverpagetopmargin+1.5in+\ht\strutbox,
  width=\textwidth - \box_wd:N \l_@@_logo_box,
  height=\box_ht:N \l_@@_logo_box,
  contents={\putUL{\box_use:N \l_@@_logo_box}},
]{title.seii.logo}
\DeclareNewPageStyleByLayers{title.seii}{title.seii.logo}
\renewcommand*{\titlepagestyle}{title.seii}
%    \end{macrocode}
%
% Redefine the \cs{maketitle} command.
% The following code is an adapted version of the corresponding \KOMAScript{}
% macro by Markus Kohm.
%    \begin{macrocode}
\renewcommand*{\maketitle}[1][1]{
  \begin{titlepage}
    \setcounter{page}{#1}
    \def\thefootnote{\fnsymbol{footnote}}
    \edef\titlepage@restore{%
      \noexpand\endgroup
      \noexpand\global\noexpand\@colht\the\@colht
      \noexpand\global\noexpand\@colroom\the\@colroom
      \noexpand\global\vsize\the\vsize
      \noexpand\global\noexpand\@titlepageiscoverpagefalse
      \noexpand\let\noexpand\titlepage@restore\noexpand\relax
    }%
    \begingroup
    \topmargin=\dimexpr \coverpagetopmargin-1in\relax
    \oddsidemargin=\dimexpr 0in\relax
    \evensidemargin=\dimexpr 0in\relax
    \textwidth=\dimexpr \paperwidth-2in\relax
    \textheight=\dimexpr
    \paperheight-\coverpagetopmargin-\coverpagebottommargin\relax
    \headheight=0pt
    \headsep=0pt
    \footskip=\baselineskip
    \@colht=\textheight
    \@colroom=\textheight
    \vsize=\textheight
    \columnwidth=\textwidth
    \hsize=\textwidth
    \linewidth=\hsize
    \setparsizes{\z@}{\z@}{\z@\@plus 1fil}\par@updaterelative
    \thispagestyle{title.seii}
    %
    \@maketitle
    %
    \if@twoside
      \@tempswatrue
      \if@tempswa
        \next@tpage
        \begin{minipage}[t]{\textwidth}
          \@uppertitleback
        \end{minipage}
        \vfill
        \begin{minipage}[b]{\textwidth}
          \@lowertitleback
        \end{minipage}\par
        \@thanks\let\@thanks\@empty
      \fi
    \fi
    \ifx\titlepage@restore\relax\else\clearpage\titlepage@restore\fi
  \end{titlepage}
}
%    \end{macrocode}
%
% \begin{macro}{\l_@@_title_box}
%   Define a box for the title if it does not yet exist.
%    \begin{macrocode}
\box_if_exist:NF \l_@@_title_box
  {
    \box_new:N \l_@@_title_box
  }
%    \end{macrocode}
% \end{macro}
%
% \begin{macro}{\@maketitle}
%   Redeclare the \cs{@maketitle} macro.
%    \begin{macrocode}
\renewcommand*{\@maketitle}{%
  \group_begin:
    \setparsizes{\z@}{\z@}{\z@\@plus 1fil}\par@updaterelative
    \thispagestyle{title.seii}
    \hbox_gset:Nn \l_@@_title_box
      {
        \parbox{\textwidth}{\__@@_print_title:}
      }
    \null
    \skip_vertical:n { 2.5 \box_ht:N \l_@@_logo_box }
    \box_use:N \l_@@_title_box
    \skip_vertical:n { .5 \box_ht:N \l_@@_logo_box }
  \group_end:
  \@thanks\let\@thanks\@empty
}
%    \end{macrocode}
% \end{macro}
%
% Load the \pkg{ifthen} package.
%    \begin{macrocode}
\RequirePackage{ifthen}
%    \end{macrocode}
%
% Prints the title formatted appropriately.
% We start with printing the title, optional subtitle, and the author names.
%    \begin{macrocode}
\cs_new:Nn \__@@_print_title:
  {
    \group_begin:
      \usekomafont{title}\centering\@title\par
    \group_end:
    \ifx\@subtitle\@empty\else{%
      \medskip\usekomafont{subtitle}\centering\@subtitle\par%
    }\fi
    \bigskip
    \group_begin:
      \usekomafont{author}\centering\@author\par
    \group_end:
    \bigskip
%    \end{macrocode}
%   The next block generates the text that describes the thesis.
%   In case of a PhD thesis, this text is predefined to match the requirements.
%   In case of a bachelor or master thesis, or a proposal thereof, we generate
%   a text based on the values of the |thesistype| load-time option and the
%   values of the \cmd{\degreeprogramme}, \cmd{\department}, and
%   \cmd{\institute} variables.
%    \begin{macrocode}
    \exp_args:NV
      {
        \begin{center}
          \tl_if_eq:NnTF \l_@@_thesis_type_tl { phd }
            {
              Dissertation~ zur~ Erlangung~ des~ Doktorgrades\\
              der~ Naturwissenschaften~ (Dr.\,rer.\,nat.)\\
              eingereicht~ an~ der~ Fakultät~ für~ Informatik~ und~ Mathematik\\
              der~ Universität~ Passau\\
              \rule{\textwidth}{.1pt}\\
              Dissertation~ submitted~ to\\
              the~ Faculty~ of~ Computer~ Science~ and~ Mathematics\\
              of~ the~ University~ of~ Passau\\
              in~ partial~ fulfillment~ of~ obtaining\\
              the~ degree~ of~ a~ Doctor~ of~ Natural~ Sciences
            } {
              \tl_if_eq:NnTF \l_@@_thesis_type_tl { phdeng }
                {
                  Dissertation~ zur~ Erlangung~ des~ Doktorgrades\\
                  der~ Ingenieurwissenschaften~ (Dr.-Ing.)\\
                  eingereicht~ an~ der~ Fakultät~ für~ Informatik~ und~ Mathematik\\
                  der~ Universität~ Passau\\
                  \rule{\textwidth}{.1pt}\\
                  Dissertation~ submitted~ to\\
                  the~ Faculty~ of~ Computer~ Science~ and~ Mathematics\\
                  of~ the~ University~ of~ Passau\\
                  in~ partial~ fulfillment~ of~ obtaining\\
                  the~ degree~ of~ a~ Doctor~ of~ Engineering
                } {
                  \tl_if_eq:NnT \l_@@_thesis_type_tl { bachelor }
                    { \GetTranslation{Bachelor-thesis} }
                  \tl_if_eq:NnT \l_@@_thesis_type_tl { bachelorproposal }
                    { \GetTranslation{Bachelor-thesis-proposal} }
                  \tl_if_eq:NnT \l_@@_thesis_type_tl { master }
                    { \GetTranslation{Master-thesis} }
                  \tl_if_eq:NnT \l_@@_thesis_type_tl { masterproposal }
                    { \GetTranslation{Master-thesis-proposal} }
                  \tl_if_eq:NnT \l_@@_thesis_type_tl { phdproposal }
                    { \GetTranslation{PhD-thesis-proposal} }
                  \tl_if_empty:NF \l_@@_degreeprogramme_tl
                    {
                      \ in~\l_@@_degreeprogramme_tl
                    }
                  \par
                  \tl_if_empty:NF \l_@@_department_tl { \l_@@_department_tl \par }
                  \tl_if_empty:NF \l_@@_institute_tl { \l_@@_institute_tl \par }
                }
            }
        \end{center}\par\bigskip
%    \end{macrocode}
%   Finally, generate a table with information about supervisors, advisors, etc.
%    \begin{macrocode}
        \begin{center}
          \begin{tabular}{@{} l @{\quad} l}
            \tl_if_empty:NF \l_@@_matrnumber_tl
              {
                \GetTranslation{Matrnumber}    & \l_@@_matrnumber_tl \\
              }
            \tl_if_empty:NF \l_@@_supervisor_tl
              {
                \GetTranslation{Supervisor}    & \l_@@_supervisor_tl \\
              }
            \tl_if_empty:NF \l_@@_cosupervisor_tl
              {
                \GetTranslation{Co-supervisor} & \l_@@_cosupervisor_tl \\
              }
            \tl_if_empty:NF \l_@@_advisor_tl
              {
                \GetTranslation{Advisor}       & \l_@@_advisor_tl \\
              }
            \tl_if_empty:NF \l_@@_coadvisor_tl
              {
                \GetTranslation{Co-advisor}    & \l_@@_coadvisor_tl \\
              }
            \tl_if_empty:NF \l_@@_external_tl
              {
                \GetTranslation{External}      & \l_@@_external_tl \\
              }
          \end{tabular}
        \end{center}
        \par\medskip
%    \end{macrocode}
%   Last, print the date or the version.
%    \begin{macrocode}
        \group_begin:
          \usekomafont{date}
          \centering
          \tl_if_empty:NTF \l_@@_version_tl
            { \@date }
            { \GetTranslation{version-of-date} }
          \par\smallskip
        \group_end:
      }
  }
%    \end{macrocode}
%
% \begin{macro}{\@lowertitleback}
%   Afterwards, override the definition of \cs{@lowertitleback}.
%    \begin{macrocode}
\renewcommand*{\@lowertitleback}{%
  \group_begin:
    \noindent\textbf{\@author}:\\
    \emph{\@title}\\
    \tl_if_eq:NnT \l_@@_thesis_type_tl { bachelor }
      { \GetTranslation{Bachelor-thesis},~ }
    \tl_if_eq:NnT \l_@@_thesis_type_tl { master }
      { \GetTranslation{Master-thesis},~ }
    \tl_if_eq:NnT \l_@@_thesis_type_tl { phd }
      { \GetTranslation{PhD-thesis},~ }
    \GetTranslation{up},~\the\year.
  \group_end:
}
%    \end{macrocode}
% \end{macro}
%
% \subsection{Styling of the sectioning commans}
%
% We change the formatting of the chapter, section, and sub-section headings.
%    \begin{macrocode}
\addtokomafont{chapterprefix}{\raggedleft\bfseries}
\addtokomafont{chapter}{\mdseries}
\addtokomafont{section}{\large\mdseries}
\addtokomafont{subsection}{\normalsize\mdseries}
\renewcommand*{\chapterformat}{%
  \enskip\mbox{%
    \scalebox{3}{\color{UPSE2-DarkGray}\thechapter\autodot}%
  }%
}
\renewcommand{\chapterlinesformat}[3]{%
  \parbox[b]{\textwidth}{\hrulefill#2}\par%
  #3\par\bigskip%
  \hrule%
}
\RedeclareSectionCommand[afterskip=1.5\baselineskip]{chapter}
\renewcommand*{\chapterlineswithprefixformat}[3]{%
  \Ifstr{#1}{chapter}
    {#2\textls*[75]{\MakeUppercase{#3}}}% change definition for chapter
    {#2#3}% original definition
}%
\renewcommand*{\sectionlinesformat}[4]{%
  #3\textls*[75]{\MakeUppercase{#4}}%
}
%    \end{macrocode}
%
% Change the loop of a \cmd{\paragraph} to be italic and end with a full stop.
%    \begin{macrocode}
\setkomafont{paragraph}{\normalsize\normalfont\itshape}
\AddtoDoHook{heading/endgroup/paragraph}{\headingdot}
\newcommand*{\headingdot}[1]{.}
%    \end{macrocode}
%
% Change the width of a dictum
%    \begin{macrocode}
\renewcommand*{\dictumwidth}{0.42\linewidth}
%    \end{macrocode}
%
% \subsection{Provide an environment for abstracts}
%
% We want to allow abstracts in German and English, which is also a requirement
% when writing a thesis in English.
% First, ensure that the \cmd{\abstract} macro is available for all classes.
% \begin{macro}{\abstract}
%    \begin{macrocode}
\providecommand{\abstract}{}
%    \end{macrocode}
% \end{macro}
% Then redefine the |abstract| environment such that it provides an optional
% argument for language selection.
% \begin{environment}{abstract}
%   Used to typeset an abstract for the thesis.
%   The optional argument allows to specify a language.
%   As a default the current document language will be used.
%    \begin{macrocode}
\RenewDocumentEnvironment { abstract } { o }
  {
    \group_begin:
      \IfNoValueF {#1} { \selectlanguage{#1} }
      \scr@ifundefinedorrelax{chapter}{
        \Iftocfeature{toc}{leveldown}
          {\subsection*}
          {\section*}
      }{
        \let\clearpage\relax
        \Iftocfeature{toc}{leveldown}
          {\section*}
          {\chapter*}
      } { \GetTranslation{Abstract} }
  } {
    \group_end:
  }
%    \end{macrocode}
% \end{environment}
%
% \subsection{Provide an environment for acknowledgements}
%
% Often, especially in a PhD thesis, one wants to acknowledge the help of
% certain people, for example, supervisors, family, and friends.
% \begin{environment}{acknowledgements}
%   Use to typeset the acknowledgements for the thesis.
%    \begin{macrocode}
\NewDocumentEnvironment { acknowledgements } { o }
  {
    \group_begin:
      \IfNoValueF {#1} { \selectlanguage{#1} }
      \scr@ifundefinedorrelax{chapter}{
        \Iftocfeature{toc}{leveldown}
          {\subsection*}
          {\section*}
      }{
        \Iftocfeature{toc}{leveldown}
          {\section*}
          {\chapter*}
      } { \GetTranslation{Acknowledgements} }
  } {
    \group_end:
  }
%    \end{macrocode}
% \end{environment}
%
% \subsection{Document structuring macros}
%
% Ensure that these macros are defined.
% These are the exact definitions from \KOMAScript.
%    \begin{macrocode}
\newif\if@mainmatter\@mainmattertrue
%    \end{macrocode}
% \begin{macro}{\frontmatter}
%   Starts the frontmatter.
%   Most notably, set the page numbers to roman.
%    \begin{macrocode}
\providecommand*{\frontmatter}
  {
    \if@twoside\cleardoubleoddpage\else\clearpage\fi
    \@mainmatterfalse
    \pagenumbering { roman }
  }
%    \end{macrocode}
% \end{macro}
% \begin{macro}{\mainmatter, \backmatter}
%   Starts the mainmatter or the backmatter.
%   Most notably, set the page numbers to arabic.
%    \begin{macrocode}
\providecommand*{\mainmatter}
  {
    \if@twoside\cleardoubleoddpage\else\clearpage\fi
    \@mainmattertrue
    \pagenumbering { arabic }
  }
\providecommand*{\backmatter}
  {
    \if@twoside\cleardoubleoddpage\else\clearpage\fi
    \@mainmatterfalse
  }
%    \end{macrocode}
% \end{macro}
%
% \subsection{Declaration of authorship}
%
% \begin{macro}{\authorshipDeclaration}
%   Provide a command to typeset the authorship declaration.
%    \begin{macrocode}
\NewDocumentCommand \authorshipDeclaration { o }
  {
    \par
    \group_begin:
      \selectlanguage{ngerman}
      \IfNoValueF {#1}
        { \tl_gset:Nn \l_@@_location_tl {#1} }

      \tl_if_empty:NT \l_@@_location_tl
        {
          \msg_new:nnn { seiithesis }
            { no-location-specified }
            {
              You~ need~ to~ specify~ a~ location~ for~ the~ authorship~
              declaration.~ Either~ via~ the~ location~ macro~ or~ via~ the~
              optional~ argument~ of~ the~ authorshipDeclaration~ macro.
            }
          \msg_warning:nn { seiithesis } { no-location-specified }
        }

      \scr@ifundefinedorrelax{chapter}{
        \Iftocfeature{toc}{leveldown}
          {\subsection*}
          {\section*}
      }{
        \Iftocfeature{toc}{leveldown}
          {\section*}
          {\chapter*}
      } { Eigenständigkeitserklärung }

      Hiermit~ versichere~ ich,~ \l_@@_signature_tl,
      \begin{enumerate}
        \item dass~ ich~ die~ vorliegende~ Arbeit~ selbstständig~ und~ ohne~
          unzulässige~ Hilfe~ verfasst~ und~ keine~ anderen~ als~ die~
          angegebenen~ Quellen~ und~ Hilfsmittel~ benutzt,~ sowie~ die~
          wörtlich~ und~ sinngemäß~ übernommenen~ Passagen~ aus~ anderen~
          Werken~ kenntlich~ gemacht~ habe.
        \item Außerdem~ erkläre~ ich,~ dass~ ich~ der~ Universität~ ein~
          einfaches~ Nutzungsrecht~ zum~ Zwecke~ der~ Überprüfung~ mittels~
          einer~ Plagiatssoftware~ in~ anonymisierter~ Form~ einräume.
      \end{enumerate}\par
      \bigskip
      \noindent \l_@@_location_tl,~ \@date\hfill
      \signatureBox{\l_@@_signature_tl}
    \group_end:
    \\\strut\cleardoublepage
  }
%    \end{macrocode}
% \end{macro}
%
% \begin{macro}{\signatureBox}
%   Provide a box for the signature.
%    \begin{macrocode}
\newcommand*{\signatureBox}[2][5cm]{
  \parbox[t]{#1}{
    \centering
    \rule{\linewidth}{.3pt}\\\makebox[0pt][c]{#2}
  }
}
%    \end{macrocode}
% \end{macro}
%
% \subsection{Research questions and summary boxes}
%
% We utilise the \pkg{ntheorem} package for defining new theorem environments
% that are used for defining research questions and hypotheses.
% Therefore, start with loading this package.
% \pkg{ntheorem}.
%    \begin{macrocode}
\slcd_package_if_loaded:nT { amsmath }
  {
    \PassOptionsToPackage{amsmath}{ntheorem}
  }
\RequirePackage{ntheorem}
\theoremseparator{:}
%    \end{macrocode}
%
% \begin{environment}{resq, hyp}
%    We can then define the \env{resq} and \env{hyp} environments using the
%    \cmd{\newtheorem} macro from \pkg{ntheorem}.
%    \begin{macrocode}
\newtheorem{resq}{Research~ Question}
\newtheorem{hyp}{Hypothesis}
%    \end{macrocode}
% \end{environment}
%
% For the summary boxes, we utilise the \pkg{tcolorbox} package.
% We start with loading this package.
%    \begin{macrocode}
\RequirePackage{tcolorbox}
%    \end{macrocode}
%
% \begin{environment}{summary}
%   The \env{summary} environment expects a label and has the summary text in
%   its content.
%    \begin{macrocode}
\NewDocumentEnvironment { summary } { m }
  { \begin{tcolorbox}[title={Summary~ (#1)}] }
  { \end{tcolorbox} }
%    \end{macrocode}
% \end{environment}
%
%
%
%
%
%
%
%
%
%    \begin{macrocode}
%</class>
%    \end{macrocode}
%
% \section{Translations for \cls{se2thesis}}\label{sec:impl-translations}
%
%    \begin{macrocode}
%<*translations>
%    \end{macrocode}
%
% \subsection{English Translations}\label{sec:impl-translations-english}
%
%    \begin{macrocode}
%<*english>
%    \end{macrocode}
%
% We provide the following English translations.
%
%    \begin{macrocode}
\ProvideDictionaryFor{English}{se2translations}[2024/02/23]
\ProvideDictTranslation{abstract}{abstract}
\ProvideDictTranslation{Abstract}{Abstract}
\ProvideDictTranslation{acknowledgement}{acknowledgement}
\ProvideDictTranslation{Acknowledgement}{Acknowledgement}
\ProvideDictTranslation{acknowledgements}{acknowledgements}
\ProvideDictTranslation{Acknowledgements}{Acknowledgements}
\ProvideDictTranslation{advisor}{advisor}
\ProvideDictTranslation{Advisor}{Advisor}
\ProvideDictTranslation{advisors}{advisors}
\ProvideDictTranslation{Advisors}{Advisors}
\ProvideDictTranslation{co-advisor}{co-advisor}
\ProvideDictTranslation{Co-advisor}{Co-advisor}
\ProvideDictTranslation{co-advisors}{co-advisors}
\ProvideDictTranslation{Co-advisors}{Co-advisors}
\ProvideDictTranslation{matrnumber}{matriculation number}
\ProvideDictTranslation{Matrnumber}{Matriculation number}
\ProvideDictTranslation{supervisor}{supervisor}
\ProvideDictTranslation{Supervisor}{Supervisor}
\ProvideDictTranslation{co-supervisor}{co-supervisor}
\ProvideDictTranslation{Co-supervisor}{Co-supervisor}
\ProvideDictTranslation{external}{external examiner}
\ProvideDictTranslation{External}{External Examiner}
\ProvideDictTranslation{degreeprogramme}{programme}
\ProvideDictTranslation{Degreeprogramme}{Programme}
\ProvideDictTranslation{Bachelor-thesis}{Bachelor Thesis}
\ProvideDictTranslation{Bachelor-thesis-proposal}{Bachelor Thesis Proposal}
\ProvideDictTranslation{Master-thesis}{Master Thesis}
\ProvideDictTranslation{Master-thesis-proposal}{Master Thesis Proposal}
\ProvideDictTranslation{PhD-thesis}{PhD Thesis}
\ProvideDictTranslation{PhD-thesis-proposal}{PhD Thesis Proposal}
\ProvideDictTranslation{date}{date}
\ProvideDictTranslation{Date}{Date}
\ProvideDictTranslation{university-of-passau}{University of Passau}
\ProvideDictTranslation{up}{University of Passau}
\ProvideDictTranslation{fim}{Faculty of Computer Science and Mathematics}
%    \end{macrocode}
%
%    \begin{macrocode}
%</english>
%    \end{macrocode}
%
% \subsection{German Translations}\label{sec:impl-translations-german}
%
%    \begin{macrocode}
%<*german>
%    \end{macrocode}
%
% We provide the following German translations.
%
%    \begin{macrocode}
\ProvideDictionaryFor{German}{se2translations}[2024/02/23]
\ProvideDictTranslation{abstract}{Zusammenfassung}
\ProvideDictTranslation{Abstract}{Zusammenfassung}
\ProvideDictTranslation{acknowledgement}{Danksagung}
\ProvideDictTranslation{Acknowledgement}{Danksagung}
\ProvideDictTranslation{acknowledgements}{Danksagungen}
\ProvideDictTranslation{Acknowledgements}{Danksagungen}
\ProvideDictTranslation{advisor}{Betreuer}
\ProvideDictTranslation{Advisor}{Betreuer}
\ProvideDictTranslation{advisors}{Betreuer}
\ProvideDictTranslation{Advisors}{Betreuer}
\ProvideDictTranslation{co-advisor}{Mitbetreuer}
\ProvideDictTranslation{Co-advisor}{Mitbetreuer}
\ProvideDictTranslation{co-advisors}{Mitbetreuer}
\ProvideDictTranslation{Co-advisors}{Mitbetreuer}
\ProvideDictTranslation{matrnumber}{Matrikelnummer}
\ProvideDictTranslation{Matrnumber}{Matrikelnummer}
\ProvideDictTranslation{supervisor}{Prüfer}
\ProvideDictTranslation{Supervisor}{Prüfer}
\ProvideDictTranslation{co-supervisor}{Zweitprüfer}
\ProvideDictTranslation{Co-supervisor}{Zweitprüfer}
\ProvideDictTranslation{external}{Externer Gutachter}
\ProvideDictTranslation{External}{Externer Gutachter}
\ProvideDictTranslation{degreeprogramme}{Studiengang}
\ProvideDictTranslation{Degreeprogramme}{Studiengang}
\ProvideDictTranslation{Bachelor-thesis}{Bachelorarbeit}
\ProvideDictTranslation{Bachelor-thesis-proposal}{Bachelorarbeitsproposal}
\ProvideDictTranslation{Master-thesis}{Masterarbeit}
\ProvideDictTranslation{Master-thesis-proposal}{Masterarbeitsproposal}
\ProvideDictTranslation{PhD-thesis}{Dissertation}
\ProvideDictTranslation{PhD-thesis-proposal}{Dissertationsproposal}
\ProvideDictTranslation{date}{Datum}
\ProvideDictTranslation{Date}{Datum}
\ProvideDictTranslation{university-of-passau}{Universität Passau}
\ProvideDictTranslation{up}{Universität Passau}
\ProvideDictTranslation{fim}{Fakultät für Informatik und Mathematik}
%    \end{macrocode}
%
%    \begin{macrocode}
%</german>
%    \end{macrocode}
%
%    \begin{macrocode}
%</translations>
%    \end{macrocode}
%
% \end{implementation}
%
% \clearpage
%
% \PrintIndex
